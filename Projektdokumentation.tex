%-----------------------------------------------------------------------------------------
% Autor dieser Vorlage:
% Stefan Macke (http://fachinformatiker-anwendungsentwicklung.net)
% Permalink zur Vorlage: http://fiae.link/LaTeXVorlageFIAE
%
% Sämtliche verwendeten Abbildungen, Tabellen und Listings stammen von Dirk Grashorn.
%
% Lizenz: Creative Commons 4.0 Namensnennung - Weitergabe unter gleichen Bedingungen
% -----------------------------------------------------------------------------------------

\documentclass[
	ngerman,
	toc=listof, % Abbildungsverzeichnis sowie Tabellenverzeichnis in das Inhaltsverzeichnis aufnehmen
	toc=bibliography, % Literaturverzeichnis in das Inhaltsverzeichnis aufnehmen
	footnotes=multiple, % Trennen von direkt aufeinander folgenden Fußnoten
	parskip=half, % vertikalen Abstand zwischen Absätzen verwenden anstatt horizontale Einrückung von Folgeabsätzen
	numbers=noendperiod % Den letzten Punkt nach einer Nummerierung entfernen (nach DIN 5008)
]{scrartcl}
\pdfminorversion=5 % erlaubt das Einfügen von pdf-Dateien bis Version 1.7, ohne eine Fehlermeldung zu werfen (keine Garantie für fehlerfreies Einbetten!)
\usepackage[utf8]{inputenc}
\usepackage[utf8]{inputenc}
\usepackage[T1]{fontenc}
\usepackage[utf8]{inputenc}
\usepackage[T1]{fontenc}
\usepackage{amsmath} % muss als erstes eingebunden werden, da Meta/Packages ggfs. Sonderzeichen enthalten

% !TEX root = Projektdokumentation.tex

% Hinweis: der Titel muss zum Inhalt des Projekts passen und den zentralen Inhalt des Projekts deutlich herausstellen
\newcommand{\titel}{Digitale Gästebegrüßung und Infotafel}
\newcommand{\untertitel}{ein Tool zur automatisierten Gästebegrüßung}
\newcommand{\kompletterTitel}{\titel{} -- \untertitel}

\newcommand{\autorName}{Jeffrey Aspinall}
\newcommand{\autorAnschrift}{Feldstraße 29}
\newcommand{\autorOrt}{38640 Goslar}

\newcommand{\betriebLogo}{logo_cstx.png}
\newcommand{\betriebName}{CSTx Enterprise Solutions GmbH}
\newcommand{\betriebAnschrift}{Volkmaroder Straße 7}
\newcommand{\betriebOrt}{38104 Braunschweig}

\newcommand{\ausbildungsberuf}{Fachinformatiker für Anwendungsentwicklung}
\newcommand{\betreff}{Dokumentation zur betrieblichen Projektarbeit}
\newcommand{\pruefungstermin}{Sommer 2024}
\newcommand{\abgabeOrt}{Braunschweig}
\newcommand{\abgabeTermin}{17.04.2024}
 % Metadaten zu diesem Dokument (Autor usw.)
\input{Allgemein/Packages} % verwendete Packages
\input{Allgemein/Seitenstil} % Definitionen zum Aussehen der Seiten
%% Abkürzungen
%\newcommand{\Versis}{\textsc{Versis}\xspace}
%\newcommand{\NI}{NatInfo\xspace}
%\newcommand{\AO}{\textsc{Alte Oldenburger} Krankenversicherung\xspace}
 % eigene allgemeine Befehle, die z.B. die Arbeit mit LaTeX erleichtern
%% Abkürzungen
%\newcommand{\Versis}{\textsc{Versis}\xspace}
%\newcommand{\NI}{NatInfo\xspace}
%\newcommand{\AO}{\textsc{Alte Oldenburger} Krankenversicherung\xspace}
 % eigene projektspezifische Befehle, z.B. Abkürzungen usw.

\begin{document}

\phantomsection
\thispagestyle{empty}
\pdfbookmark[1]{Eidesstattliche Erklärung}{ihkdeckblatt}
\includegraphicsKeepAspectRatio{DeckblattIHK}{1}
\cleardoublepage

\phantomsection
\thispagestyle{plain}
\pdfbookmark[1]{Deckblatt}{deckblatt}
\input{Deckblatt}
\cleardoublepage

% Preface --------------------------------------------------------------------
\phantomsection
\pagenumbering{Roman}
\pdfbookmark[1]{Inhaltsverzeichnis}{inhalt}
\tableofcontents

\cleardoublepage

\phantomsection
\listoffigures
\cleardoublepage

\phantomsection
\listoftables
\cleardoublepage

\phantomsection
\lstlistoflistings
\cleardoublepage

\newcommand{\abkvz}{Abkürzungsverzeichnis}
\renewcommand{\nomname}{\abkvz}
\section*{\abkvz}
\markboth{\abkvz}{\abkvz}
\addcontentsline{toc}{section}{\abkvz}
% !TEX root = Projektdokumentation.tex

% Es werden nur die Abkürzungen aufgelistet, die mit \ac definiert und auch benutzt wurden. 
%
% \acro{VERSIS}{Versicherungsinformationssystem\acroextra{ (Bestandsführungssystem)}}
% Ergibt in der Liste: VERSIS Versicherungsinformationssystem (Bestandsführungssystem)
% Im Text aber: \ac{VERSIS} -> Versicherungsinformationssystem (VERSIS)

% Hinweis: allgemein bekannte Abkürzungen wie z.B. bzw. u.a. müssen nicht ins Abkürzungsverzeichnis aufgenommen werden
% Hinweis: allgemein bekannte IT-Begriffe wie Datenbank oder Programmiersprache müssen nicht erläutert werden,
%          aber ggfs. Fachbegriffe aus der Domäne des Prüflings (z.B. Versicherung)

% Die Option (in den eckigen Klammern) enthält das längste Label oder
% einen Platzhalter der die Breite der linken Spalte bestimmt.
\begin{acronym}[WWWWW]
	\acro{AD}{Active Directory}
	\acro{AG}{Aktiengesellschaft}
	\acro{API}{Application Programming Interface}
	\acro{CSS}{Cascading Style Sheets}
	\acro{CSTx ES}{CSTx Enterprise Solutions}
	\acro{CSTx SE}{CSTx Software Engineering}
	\acro{CP}{Container-Presentational}
	\acro{DOM}{Document Object Model}
	\acro{GL}{GitLab}
	\acro{GmbH}{Gesellschaft mit beschränkter Haftung}
	\acro{HTML}{Hypertext Markup Language}
	\acro{IDE}{Integrated Development Environment}
	\acro{IHK}{Industrie- und Handelskammer}
	\acro{IPC}{Inter-Process Communication}
	\acro{IV}{Informationsverteilung}
	\acro{JS}{JavaScript}
	\acro{JSX}{JavaScript XML}
	\acro{MS}{Microsoft}
	\acro{MSAL}{Microsoft Authentication Library}
	\acro{SV}{Source Versioning}
	\acro{TS}{TypeScript}
	\acro{TSX}{TypeScript XML}
	\acro{UI}{User Interface}
	\acro{UML}{Unified Modeling Language}
	\acro{VS}{Visual Studio}
\end{acronym}

\clearpage

% Inhalt ---------------------------------------------------------------------
\pagenumbering{arabic}
% !TEX root = Projektdokumentation.tex
% !TEX root = ../Projektdokumentation.tex
\section{Einleitung}
\label{sec:Einleitung}


\subsection{Projektumfeld} 
\label{sec:Projektumfeld}
%\begin{itemize}
%	\item Kurze Vorstellung des Ausbildungsbetriebs (Geschäftsfeld, Mitarbeiterzahl \usw)
%	\item Wer ist Auftraggeber/Kunde des Projekts?
%\end{itemize}
Seit 2005 gibt es \ac{CSTx SE} und im Jahr 2008 wurde \ac{CSTx SE} zum Umsetzungspartner für Volkswagen \ac{AG}.
Durch den Großkunden konnte sich die \ac{CSTx SE} in der Automobilbranche einen festen Platz sichern.
Im Jahr 2018 gab es eine Umstrukturierung der Holding und die Gründung der drei Tochterunternehmen kam zustande.
Insgesamt ist \ac{CSTx SE} ein Mittelständiger Betrieb mit einer Mannschaftsgröße von knapp 100 Mitarbeitern.
Die Ausbildung und die Projektarbeit findet in der Tochterfirma \ac{CSTx ES} statt.
Das Ergebnis des Projekts muss dem Stakeholder präsentiert werden, der das Projekt initiiert hat und Anwender sein wird, der die Anwendung vorrangig nutzt.


\subsection{Projektziel} 
\label{sec:Projektziel}
%\begin{itemize}
%	\item Worum geht es eigentlich?
%	\item Was soll erreicht werden?
%\end{itemize}
Es soll eine innovative Lösung für die Gästebegrüßung und den Informationsfluss entwickelt werden. 
Dies umfasst die Darstellung einer auf den Kunden angepasste Info-Karte, die Integration von Live-Video-Streams, einem laufenden Info-Text und einer Uhrzeitangabe. 
Durch diese Elemente sollen Gäste effektiv und ansprechend begrüßt werden und gleichzeitig wichtige Informationen erhalten. 
Das Hauptziel besteht darin, ein visuell ansprechendes und informatives Tool zu schaffen, das die Aufmerksamkeit der Besucher auf sich zieht und einen positiven ersten Eindruck vermittelt.


\subsection{Projektbegründung} 
\label{sec:Projektbegruendung}
%\begin{itemize}
%	\item Warum ist das Projekt sinnvoll (\zB Kosten- oder Zeitersparnis, weniger Fehler)?
%	\item Was ist die Motivation hinter dem Projekt?
%\end{itemize}
Es besteht eine Notwendigkeit einer zeitgemäßen und effizienten Gästebegrüßung sowie Informationsbereitstellung.
Das Unternehmen strebt nach einem Produkt, das Prozesse für den Informationsaustausch vereinfacht und beschleunigt.
Zudem soll es äußerst flexibel anpassbar sein und sowohl für Kunden als auch Mitarbeiter einen Mehrwert bieten.
Durch die Digitalisierung der Begrüßungs- und Informationsprozesse soll ein innovatives und professionelles Ambiente geschaffen werden, das die Attraktivität der Einrichtung steigert und die Zufriedenheit der Besucher erhöht.


\subsection{Projektschnittstellen} 
\label{sec:Projektschnittstellen}
%\begin{itemize}
%	\item Mit welchen anderen Systemen interagiert die Anwendung (technische Schnittstellen)?
%	\item Wer genehmigt das Projekt \bzw stellt Mittel zur Verfügung?
%	\item Wer sind die Benutzer der Anwendung?
%	\item Wem muss das Ergebnis präsentiert werden?
%\end{itemize}
Die Anwendung interagiert mit \ac{MS} Azure \ac{AD} für die Authentifizierung und den Zugriff auf Kalendereinträge.
Die Genehmigung des Projekts sowie die Bereitstellung von Ressourcen erfolgt durch den Stakeholder und Entscheidungsträger, wie \bspw die die Geschäftsführung.
Jeder Mitarbeiter, der über einen Azure-Account der Organisation verfügt, kann sich in die Anwendung einloggen und diese nutzen.
Die Anwendung zeigt die kommenden Kalendereinträge des Tages an und kann somit auch als Dashboard für den kommenden Arbeitstag einzelner Mitarbeiter verwendet werden.

\subsection{Projektabgrenzung} 
\label{sec:Projektabgrenzung}
%\begin{itemize}
%	\item Was ist explizit nicht Teil des Projekts (\insb bei Teilprojekten)?
%\end{itemize}
Die Einrichtung der Authentifizierung seitens Azure \ac{AD} sowie das Deployment sind nicht Teil des Projekts.
Darüber hinaus wird ausdrücklich gewünscht, dass keine Datenbank verwendet wird, da lediglich das Auslesen der Kalendereinträge erforderlich ist.
Die Einbindung weiterer Integrationen von Authentifizierungs- und Kalendersupport von \zB Google, ist nicht Teil des Projekts.
% !TEX root = ../Projektdokumentation.tex
\section{Projektplanung} 
\label{sec:Projektplanung}


\subsection{Projektphasen}
\label{sec:Projektphasen}

%\begin{itemize}
%	\item In welchem Zeitraum und unter welchen Rahmenbedingungen (\zB Tagesarbeitszeit) findet das Projekt statt?
%	\item Verfeinerung der Zeitplanung, die bereits im Projektantrag vorgestellt wurde.
%\end{itemize}
Das Projekt startet am 26. Februar 2024 und endet am 12. April 2024, wobei es in Teilzeit durchgeführt wird.
Während dieser Zeit wird es im Büro bearbeitet, um eine effektive Zusammenarbeit mit Mark Barrenscheen zu gewährleisten.
Die Nähe ermöglicht es, Änderungen schnell zu besprechen und umzusetzen.

Tabelle~\ref{tab:Zeitplanung} zeigt eine grobe Zeitplanung.
\tabelle{Zeitplanung}{tab:Zeitplanung}{ZeitplanungKurz}\\
Eine detailliertere Zeitplanung findet sich im \Anhang{app:Zeitplanung}.


\subsection{Abweichungen vom Projektantrag}
\label{sec:AbweichungenProjektantrag}
%\begin{itemize}
%	\item Sollte es Abweichungen zum Projektantrag geben (\zB Zeitplanung, Inhalt des Projekts, neue Anforderungen), müssen diese explizit aufgeführt und begründet werden.
%\end{itemize}
In diesem Projekt wurden keine automatisierten Tests eingesetzt, da die Anwendung eine einfache Funktionalität aufweist, die primär auf das Auslesen von Kalendereinträgen ausgerichtet ist.
Die Komplexität der Funktionalität sowie die Anforderungen an die Robustheit und Skalierbarkeit der Anwendung waren nicht hoch genug, um den Einsatz automatisierter Tests zu rechtfertigen.
Zudem wäre der Aufwand für die Implementierung automatisierter Tests im Vergleich zum potenziellen Nutzen möglicherweise nicht verhältnismäßig gewesen.
Stattdessen wurde auf manuelle Tests gesetzt, um sicherzustellen, dass die grundlegende Funktionalität der Anwendung ordnungsgemäß funktioniert und den Anforderungen entspricht.


\subsection{Ressourcenplanung}
\label{sec:Ressourcenplanung}

%\begin{itemize}
%	\item Detaillierte Planung der benötigten Ressourcen (Hard-/Software, Räumlichkeiten \usw).
%	\item \Ggfs sind auch personelle Ressourcen einzuplanen (\zB unterstützende Mitarbeiter).
%\end{itemize}
Es werden ein Arbeitsgerät, eine stabile Internetverbindung sowie eine geeignete Entwicklungsumgebung benötigt, hier wurde dafür VSCode als IDE verwendet.
Da die Anwendung auf Microsoft Azure Active Directory zugriff, war der Zugang zu Azure AD unerlässlich.
Zusätzlich wurde im Büro ein Fernseher installiert, um Präsentationen und Demos der Anwendung zu ermöglichen.
Für die Remote-Arbeit wurden Remote-Tastatur, Maus und ein Mini-PC bereitgestellt, um eine reibungslose Zusammenarbeit zwischen den Teammitgliedern zu gewährleisten.
Diese sorgfältige Ressourcenplanung trug dazu bei, dass das Team effizient arbeiten konnte und Mark Barrenscheen sowie andere Nutzer die Anwendung produktiv nutzen konnten.


\subsection{Entwicklungsprozess}
\label{sec:Entwicklungsprozess}
\begin{itemize}
	\item Welcher Entwicklungsprozess wird bei der Bearbeitung des Projekts verfolgt (\zB Wasserfall, agiler Prozess)?
\end{itemize}

% !TEX root = ../Projektdokumentation.tex
\section{Analysephase} 
\label{sec:Analysephase}


\subsection{Ist-Analyse} 
\label{sec:IstAnalyse}
%\begin{itemize}
%	\item Wie ist die bisherige Situation (\zB bestehende Programme, Wünsche der Mitarbeiter)?
%	\item Was gilt es zu erstellen/verbessern?
%\end{itemize}
Bisher müssen Kunden auf ihren Ansprechpartner im Flur warten und das Unternehmen wünscht sich eine digitale Lösung, diese automatisiert zu begrüßen und vorab über ihren Termin zu informieren.


\subsection{Wirtschaftlichkeitsanalyse}
\label{sec:Wirtschaftlichkeitsanalyse}
%\begin{itemize}
%	\item Lohnt sich das Projekt für das Unternehmen?
%\end{itemize}
Dieses Projekt bietet eine Vielzahl von Vorteile.
Durch die Automatisierung der Gästebegrüßung und Informationsbereitstellung wird die Effizienz im Eingangsbereich erheblich gesteigert.
%Dies optimiert nicht nur die Arbeitsabläufe, sondern hinterlässt auch einen positiven ersten Eindruck bei den Gästen und könnte möglicherweise ihre Zufriedenheit steigern.
%Dadurch könnte eine stärkere Kundenbindung erreicht werden.
Basierend auf den potenziellen Effizienzsteigerungen im Eingangsbereich sowie der Möglichkeit, einen positiven Eindruck bei den Gästen zu hinterlassen und deren Zufriedenheit zu steigern, scheint das Projekt eine lohnende Investition zu sein.


\subsubsection{\gqq{Make or Buy}-Entscheidung}
\label{sec:MakeOrBuyEntscheidung}
%\begin{itemize}
%	\item Gibt es vielleicht schon ein fertiges Produkt, dass alle Anforderungen des Projekts abdeckt?
%	\item Wenn ja, wieso wird das Projekt trotzdem umgesetzt?
%\end{itemize}
Das Unternehmen entscheidet sich für die interne Entwicklung, da es sicherstellen will, dass das Design genau seinen Vorstellungen entspricht und die Informationen präzise auf seine Bedürfnisse zugeschnitten sind.
Durch die interne Entwicklung behält das Unternehmen die volle Kontrolle über den Prozess, was die geforderte Flexibilität ermöglicht, um auf Änderungen zu reagieren.


\subsubsection{Projektkosten}
\label{sec:Projektkosten}
%\begin{itemize}
%	\item Welche Kosten fallen bei der Umsetzung des Projekts im Detail an (\zB Entwicklung, Einführung/Schulung, Wartung)?
%\end{itemize}

\paragraph{Rechnung (verkürzt)}
Die Kosten für die Durchführung des Projekts setzen sich sowohl aus Personal-, als auch aus Ressourcenkosten zusammen.
Laut Tarifvertrag verdient ein Auszubildender im dritten Lehrjahr pro Monat \eur{900} Brutto.

\begin{eqnarray}
8 \mbox{ h/Tag} \cdot 220 \mbox{ Tage/Jahr} = 1.760 \mbox{ h/Jahr}\\
\eur{900}\mbox{/Monat} \cdot 13,3 \mbox{ Monate/Jahr} = \eur{11.970} \mbox{/Jahr}\\
\frac{\eur{11.970} \mbox{/Jahr}}{1760 \mbox{ h/Jahr}} \approx \eur{6,81}\mbox{/h}
\end{eqnarray}

Es ergibt sich also ein Stundenlohn von \eur{6,81}.
Die Durchführungszeit des Projekts beträgt 80 Stunden. Für die Nutzung von Ressourcen\footnote{Räumlichkeiten, Arbeitsplatzrechner etc.} wird
ein pauschaler Stundensatz von \eur{14} angenommen. Für die anderen Mitarbeiter wird pauschal ein Stundenlohn von \eur{25} angenommen.
Eine Aufstellung der Kosten befindet sich in Tabelle~\ref{tab:Kostenaufstellung} und sie betragen insgesamt \eur{1.859,80}.
\tabelle{Kostenaufstellung}{tab:Kostenaufstellung}{Kostenaufstellung.tex}


\subsubsection{Amortisationsdauer}
\label{sec:Amortisationsdauer}
%\begin{itemize}
%	\item Welche monetären Vorteile bietet das Projekt (\zB Einsparung von Lizenzkosten, Arbeitszeitersparnis, bessere Usability, Korrektheit)?
%	\item Wann hat sich das Projekt amortisiert?
%\end{itemize}
Das Produkt führt voraussichtlich zu einer Verkürzung der Vorbereitungsdauer eines Termins um etwa 20 Minuten.
Für zusätzliche Zeitersparnisse durch eine optimierte Informationsverteilung werden pauschal 30 Minuten einkalkuliert.
Angenommen wird, dass durchschnittlich 10 Kunden zu Besuch kommen und sich 3 Mal pro Woche allgemeine Informationen ändern, während 1 Mitarbeiter dafür verantwortlich ist.
Unter Berücksichtigung dieser Annahmen ergibt sich folgende Berechnung.

\paragraph{Rechnung (verkürzt)}
Bei einem Zeitersparnis von 20 Minuten pro Termin und 30 Minuten für jede Informationsverteilung, bei einem Benutzer und 220 Arbeitstagen im Jahr, wobei sich 10 Termine und 3 Informationsverteilungen pro Woche ereignen, ergibt sich folgendes Gesamtzeitersparnis.
\begin{eqnarray}
220 \mbox{ Tage/Jahr} \cdot (20 \mbox{ min/Termin} \cdot 10 \mbox{ mal/Woche}) = 8800 \mbox{ min/Jahr} \approx 147 \mbox{ h/Jahr}\\
220 \mbox{ Tage/Jahr} \cdot (30 \mbox{ min/IV} \cdot 3 \mbox{ mal/Woche}) = 3960 \mbox{ min/Jahr} \approx 66 \mbox{ h/Jahr}
\end{eqnarray}

Dadurch ergibt sich eine jährliche Einsparung von 
\begin{eqnarray}
(147 \mbox{ h} + 66 \mbox{ h}) \cdot \eur{(25 + 14)}{\mbox{/h}} = \eur{8307}
\end{eqnarray}

Die Amortisationszeit beträgt also $\frac{\eur{1.859,80}}{\eur{8307}\mbox{/Jahr}} \approx 0,02 \mbox{ Jahre} \approx 1 \mbox{ Woche}$.


\subsection{Nutzwertanalyse}
\label{sec:Nutzwertanalyse}
\begin{itemize}
	\item Darstellung des nicht-monetären Nutzens (\zB Vorher-/Nachher-Vergleich anhand eines Wirtschaftlichkeitskoeffizienten). 
\end{itemize}

\paragraph{Beispiel}
Ein Beispiel für eine Entscheidungsmatrix findet sich in Kapitel~\ref{sec:Architekturdesign}: \nameref{sec:Architekturdesign}.


\subsection{Anwendungsfälle}
\label{sec:Anwendungsfaelle}
\begin{itemize}
	\item Welche Anwendungsfälle soll das Projekt abdecken?
	\item Einer oder mehrere interessante (!) Anwendungsfälle könnten exemplarisch durch ein Aktivitätsdiagramm oder eine \ac{EPK} detailliert beschrieben werden. 
\end{itemize}

\paragraph{Beispiel}
Ein Beispiel für ein Use Case-Diagramm findet sich im \Anhang{app:UseCase}.


\subsection{Qualitätsanforderungen}
\label{sec:Qualitaetsanforderungen}
\begin{itemize}
	\item Welche Qualitätsanforderungen werden an die Anwendung gestellt (\zB hinsichtlich Performance, Usability, Effizienz \etc (siehe \citet{ISO9126}))?
\end{itemize}


\subsection{Lastenheft/Fachkonzept}
\label{sec:Lastenheft}
\begin{itemize}
	\item Auszüge aus dem Lastenheft/Fachkonzept, wenn es im Rahmen des Projekts erstellt wurde.
	\item Mögliche Inhalte: Funktionen des Programms (Muss/Soll/Wunsch), User Stories, Benutzerrollen
\end{itemize}

\paragraph{Beispiel}
Ein Beispiel für ein Lastenheft findet sich im \Anhang{app:Lastenheft}. 

% !TEX root = ../Projektdokumentation.tex
\section{Vorbereitungsphase}
\label{sec:Vorbereitungsphase}

\subsection{Zielplattform}
\label{sec:Zielplattform}
%\begin{itemize}
%	\item Beschreibung der Kriterien zur Auswahl der Zielplattform (\ua Programmiersprache, Datenbank, Client/Server, Hardware).
%\end{itemize}
In Bezug auf die Programmiersprache und das Framework fiel die Wahl auf \ac{TS} in Verbindung mit dem React-Framework.
\ac{TS} bietet die Möglichkeit, eine robuste und wartbare Codebasis zu entwickeln, während React eine interaktive Benutzeroberfläche ermöglicht.
Diese Kombination erleichtert die Entwicklung einer benutzerfreundlichen Anwendung und bietet die Möglichkeit zur Wiederverwendung von Code.


Die Entscheidung für die Laufzeitumgebung fiel auf Electron, um eine plattformübergreifende Kompatibilität sicherzustellen und die Nutzung von Webtechnologien zu ermöglichen.
Durch Electron kann die Anwendung als Desktop-App ausgeführt werden und ist somit auf verschiedenen Betriebssystemen wie Windows, macOS und Linux nutzbar.


\subsection{Architekturdesign}
\label{sec:Architekturdesign}
%\begin{itemize}
%	\item Beschreibung und Begründung der gewählten Anwendungsarchitektur (\zB \acs{MVC}).
%	\item \Ggfs Bewertung und Auswahl von verwendeten Frameworks sowie \ggfs eine kurze Einführung in die Funktionsweise des verwendeten Frameworks.
%\end{itemize}
Die gewählte Architektur basiert auf dem Konzept des \ac{CP}-Musters, das eine klare Trennung zwischen der Logik der Anwendung und der Darstellung der Benutzeroberfläche ermöglicht.
Dieses Muster wird oft in Kombination mit dem React-Framework verwendet und ermöglicht eine effiziente Entwicklung und Wartung der Anwendung


Im Rahmen dieser Architektur werden die verschiedenen Komponenten der Anwendung in Container-Komponenten und Präsentationskomponenten aufgeteilt.
Die Container-Komponenten sind für die Logik der Anwendung zuständig, während die Präsentationskomponenten die Darstellung der Benutzeroberfläche übernehmen.
Dadurch wird eine klare Trennung von Daten und Darstellung erreicht, was die Wiederverwendbarkeit und Testbarkeit der Komponenten verbessert.


Die Wahl des React-Frameworks unterstützt diese Architektur, da React die Entwicklung von wiederverwendbaren \ac{UI}-Komponenten erleichtert und eine effiziente Aktualisierung der Benutzeroberfläche ermöglicht.
React verwendet eine virtuelle \ac{DOM}, um Änderungen effizient zu verarbeiten und die Benutzeroberfläche schnell zu aktualisieren.

%\paragraph{Beispiel}
%Anhand der Entscheidungsmatrix in Tabelle~\ref{tab:Entscheidungsmatrix} wurde für die Implementierung der Anwendung das \acs{PHP}-Framework Symfony\footnote{\Vgl \citet{Symfony}.} ausgewählt.
%
%\tabelle{Entscheidungsmatrix}{tab:Entscheidungsmatrix}{Nutzwert.tex}


\subsection{Entwurf der Benutzeroberfläche}
\label{sec:Benutzeroberflaeche} 
%\begin{itemize}
%	\item Entscheidung für die gewählte Benutzeroberfläche (\zB GUI, Webinterface).
%	\item Beschreibung des visuellen Entwurfs der konkreten Oberfläche (\zB Mockups, Menüführung).
%	\item \Ggfs Erläuterung von angewendeten Richtlinien zur Usability und Verweis auf Corporate Design.
%\end{itemize}
Die Benutzeroberfläche besteht aus vier Hauptkomponenten, die auf einen Blick alle relevanten Informationen anzeigen: die Uhrzeit und das Datum, den Live-Video-Feed, die Laufschrift und die Terminauflistung.
Die Anordnung der Komponenten erfolgt so, dass sie übersichtlich und leicht erkennbar sind.

Die Uhrzeit und das Datum werden prominent oben rechts oder oben links auf der Anzeige platziert, um sie leicht sichtbar zu machen.
Dies ermöglicht den Besuchern, jederzeit die aktuelle Zeit und das Datum im Blick zu haben.

Der Live-Video-Feed nimmt den Bereich oben links oder unten links der Benutzeroberfläche ein.
Dadurch können sie wichtige Ereignisse oder Informationen in Echtzeit verfolgen.

Die Laufschrift wird am unteren Rand der Anzeige positioniert und zeigt kontinuierlich wichtige Nachrichten oder Mitteilungen von oder über das Unternehmen an.
Dies ermöglicht es den Besuchern, relevante Informationen schnell zu erfassen, selbst wenn sie sich nicht aktiv auf die Anzeige konzentrieren.

Die Terminauflistung wird angezeigt, wenn Termine vorhanden sind, und zeigt eine Liste der kommenden Ereignisse oder Besprechungen über die gesamte rechte Seite an.
Wenn keine Termine vorhanden sind, wird die Terminauflistung ausgeblendet, um Platz für die anderen Komponenten zu schaffen.
Termine innerhalb der Auflistung werden genutzt um letzlich den Besucher zu begrüßen.

Es gibt kein traditionelles Menü oder \ac{UI} in der Anwendung.
Die Anwendung wird ausschließlich von Hotkeys gesteuert, die von den Benutzern verwendet werden können, um zwischen den verschiedenen Ansichten zu navigieren oder Aktionen auszuführen.
Darüber hinaus reagiert die Anwendung auf Kalenderupdates, um Änderungen in den Terminen sofort zu reflektieren und die Anzeige entsprechend anzupassen.

\paragraph{Wireframes}
Entwürfe für die Oberfläche finden sich im \Anhang{app:Entwuerfe}.


%\subsection{Datenmodell}
%\label{sec:Datenmodell}
%
%\begin{itemize}
%	\item Entwurf/Beschreibung der Datenstrukturen (\zB \acs{ERM} und/oder Tabellenmodell, \acs{XML}-Schemas) mit kurzer Beschreibung der wichtigsten (!) verwendeten Entitäten.
%\end{itemize}
%
%\paragraph{Beispiel}
%In \Abbildung{ER} wird ein \ac{ERM} dargestellt, welches lediglich Entitäten, Relationen und die dazugehörigen Kardinalitäten enthält.
%
%\begin{figure}[htb]
%\centering
%\includegraphicsKeepAspectRatio{ERDiagramm.pdf}{0.6}
%\caption{Vereinfachtes ER-Modell}
%\label{fig:ER}
%\end{figure}


\subsection{Geschäftslogik}
\label{sec:Geschaeftslogik}
%\begin{itemize}
%	\item Modellierung und Beschreibung der wichtigsten (!) Bereiche der Geschäftslogik (\zB mit Kom\-po\-nen\-ten-, Klassen-, Sequenz-, Datenflussdiagramm, Programmablaufplan, Struktogramm, \ac{EPK}).
%	\item Wie wird die erstellte Anwendung in den Arbeitsfluss des Unternehmens integriert?
%\end{itemize}

Eine konkrete Geschäftslogik ist nicht Teil des Projekts und wird von \ac{MS} Azure \ac{AD} genutzt.
Es wird lediglich das benötigte Anfordern und die dazugehörige Darstellung von Kontakt und Termindaten implementiert.

Dieses Projekts ist klar strukturiert und ist unterteilt zwischen einer Renderer- und Main-Komponente.
Hier wird das Electron-Framework verwendet, um sowohl eine Chromium Instanz als auch die Kommunikation zwischen den beiden Komponente mittels des Observer-Patterns bereitzustellen.
Der Renderer nutzt React, \ac{TS} und Tailwind \ac{CSS}, um eine interaktive und ansprechende Oberfläche zu gestalten und integriert zusätzlich einen Key-Handler für eine vereinfachte Benutzerinteraktion.
Innerhalb des Renderers sind verschiedene visuelle Komponenten wie Uhrzeit und Datum, Live-Video, Laufschrift, eine auflistung von Terminen und Login-Funktion integriert.

Die Main-Komponente ist für die Verwaltung von Fenstern, Stores und Utilities verantwortlich und nutzt die Integration von Helper-Funktionen.
Die Authentifizierung erfolgt über \ac{MSAL}, während eine \ac{MS} Graph-Schnittstelle über Axios implementiert wird, um auf die Kalendereinträge von dem Benutzer zuzugreifen.

\paragraph{Diagramm}
Eine vereinfachte Darstellung der Logik ist im \Anhang{app:Projektstruktur} zu finden.

%\Abbildung{Modulimport} zeigt den grundsätzlichen Programmablauf beim Einlesen eines Moduls als \ac{EPK}.
%\begin{figure}[htb]
%\centering
%\includegraphicsKeepAspectRatio{modulimport.pdf}{0.9}
%\caption{Prozess des Einlesens eines Moduls}
%\label{fig:Modulimport}
%\end{figure}


\subsection{Maßnahmen zur Qualitätssicherung}
\label{sec:Qualitaetssicherung}
%\begin{itemize}
%	\item Welche Maßnahmen werden ergriffen, um die Qualität des Projektergebnisses (siehe Kapitel~\ref{sec:Qualitaetsanforderungen}: \nameref{sec:Qualitaetsanforderungen}) zu sichern (\zB automatische Tests, Anwendertests)?
%	\item \Ggfs Definition von Testfällen und deren Durchführung (durch Programme/Benutzer).
%\end{itemize}
Um die Qualität des Projektergebnisses sicherzustellen, werden ausschließlich Anwendertests durchgeführt.
Diese Tests werden von dem Entwickler und tatsächlichen Benutzern der Anwendung durchgeführt, um sicherzustellen, dass sie den Anforderungen und Erwartungen des Stakeholders entsprechen.

Die Anwendertests umfassen eine Reihe von Szenarien und Aufgaben, die die Benutzer durchführen sollen.
Diese können \bspw das Live-Video Signal durch die Key-Events wechseln oder einen 0 Uhr Kalendereintrag setzen, um die Konfiguration zu testen.

Es liegt der Fokus darauf, die Anwendung in einer realen Umgebung zu testen und Feedback von den Benutzern einzuholen.
Dies ermöglicht es, potenzielle Probleme oder Verbesserungsmöglichkeiten frühzeitig zu identifizieren und direkt in die Entwicklung einzubeziehen.

Die Anwendertests werden kontinuierlich während des Entwicklungsprozesses durchgeführt, somit wird die Anforderungen und die Qualität des Projektergebnisses gewährleistet.


%\subsection{Pflichtenheft/Datenverarbeitungskonzept}
%\label{sec:Pflichtenheft}
%\begin{itemize}
%	\item Auszüge aus dem Pflichtenheft/Datenverarbeitungskonzept, wenn es im Rahmen des Projekts erstellt wurde.
%\end{itemize}
%
%\paragraph{Beispiel}
%Ein Beispiel für das auf dem Lastenheft (siehe Kapitel~\ref{sec:Lastenheft}: \nameref{sec:Lastenheft}) aufbauende Pflichtenheft ist im \Anhang{app:Pflichtenheft} zu finden.

% !TEX root = ../Projektdokumentation.tex
\section{Implementierungsphase} 
\label{sec:Implementierungsphase}

%\subsection{Implementierung der Datenstrukturen}
%\label{sec:ImplementierungDatenstrukturen}
%
%\begin{itemize}
%	\item Beschreibung der angelegten Datenbank (\zB Generierung von \acs{SQL} aus Modellierungswerkzeug oder händisches Anlegen), \acs{XML}-Schemas \usw.
%\end{itemize}


\subsection{Implementierung der Benutzeroberfläche}
\label{sec:ImplementierungBenutzeroberflaeche}

%\begin{itemize}
%	\item Beschreibung der Implementierung der Benutzeroberfläche, falls dies separat zur Implementierung der Geschäftslogik erfolgt (\zB bei \acs{HTML}-Oberflächen und Stylesheets).
%	\item \Ggfs Beschreibung des Corporate Designs und dessen Umsetzung in der Anwendung.
%	\item Screenshots der Anwendung
%\end{itemize}
Die Implementierung der Benutzeroberfläche erfolgt separat von der Implementierung der Geschäftslogik.
In diesem Fall wird das React-Framework verwendet, um die verschiedenen \ac{UI}-Komponenten zu entwickeln und zu gestalten.

Die Benutzeroberfläche wird mithilfe von \ac{TSX} erstellt, einer Kombination aus \ac{TS} und \ac{JSX}, welches einer Syntaxerweiterung von \ac{JS} ist, die es ermöglicht, \ac{HTML}-ähnliche Elemente direkt in \ac{JS}-Code einzubetten.
Dadurch können die verschiedenen \ac{UI}-Komponenten in einer klaren und strukturierten Weise erstellt werden, wobei die Prozesse zur formatierung und das Markup eng miteinander verbunden sind.

Für das Styling wird Tailwind \ac{CSS}, ein Utility-First \ac{CSS}-Framework, das eine Reihe von vorgefertigten Utility-Klassen bereitstellt, genutzt.
Dadurch wird die Entwicklung von \ac{CSS} stark vereinfacht und beschleunigt, da keine separaten Stylesheets erstellt werden müssen.

\paragraph{Produkt in Aktion}
Bilder vom Produkt im Eingangsbereich mit Dummy-Daten befinden sich im \Anhang{Produktfotos}.


\subsection{Implementierung der Geschäftslogik}
\label{sec:ImplementierungGeschaeftslogik}
%\begin{itemize}
%	\item Beschreibung des Vorgehens bei der Umsetzung/Programmierung der entworfenen Anwendung.
%	\item \Ggfs interessante Funktionen/Algorithmen im Detail vorstellen, verwendete Entwurfsmuster zeigen.
%	\item Quelltextbeispiele zeigen.
%	\item Hinweis: Wie in Kapitel~\ref{sec:Einleitung}: \nameref{sec:Einleitung} zitiert, wird nicht ein lauffähiges Programm bewertet, sondern die Projektdurchführung. Dennoch würde ich immer Quelltextausschnitte zeigen, da sonst Zweifel an der tatsächlichen Leistung des Prüflings aufkommen können.
%\end{itemize}
Die Geschäftslogik umfasst mehrere Schlüsselfunktionen wie das Bereitstellen von Kalenderdaten wofür die Kommunikation mit \ac{MS} Azure \ac{AD} benötigt wird.
Zusätzlich wird die Verarbeitung von Benutzerinteraktionen und die Verwaltung von Zuständen Bestandteil der Geschäftslogik.


Für die Bereitstellung von Kalenderdaten ist es wichtig, Daten aus \ac{MS} Azure \ac{AD} abzurufen, sie zu parsen und entsprechend vorzubereiten, um kommende Termine und Veranstaltungen im Renderer anzuzeigen.
Diese Daten müssen dynamisch in die Benutzeroberfläche integriert werden, um den Benutzern einen aktuellen Überblick zu bieten, dies wird erreicht, indem eine \ac{IPC} basierend auf dem Observer-Pattern erstellt wird.


Die Verwaltung von Benutzerinteraktionen ist ein weiterer zentraler Aspekt der Geschäftslogik.
Da die Anwendung von einer aktiven Authentifikation abhängig ist, müssen Funktionalitäten definiert und entsprechende Aktionen zugewiesen werden, die den Authorization Code Flow von \ac{MS} Azure \ac{AD} nutzen.
Dies erfordert das Implementieren von \ac{MSAL} und einer korrekten handhabung von Access-Token.


Die Verwaltung von Zuständen ist ein entscheidender Aspekt, um sicherzustellen, dass die Benutzeroberfläche der Anwendung korrekt aktualisiert wird und auf Änderungen reagiert.
Dies umfasst, das Definieren und Aktualisieren von Konfigurationen und der zu nutzenden Session, sowie das Aktualisieren vom aktuellen Zustand im Renderer.


\paragraph{Teile der Implementierung}
Der Logik-Loop für das Abrufen der Kalendereinträge in der \texttt{Main\-Kom\-po\-nen\-te} findet sich im \Anhang{app:MAINLOOP}.

% !TEX root = ../Projektdokumentation.tex
\section{Abnahmephase} 
\label{sec:Abnahmephase}

%\begin{itemize}
%	\item Welche Tests (\zB Unit-, Integrations-, Systemtests) wurden durchgeführt und welche Ergebnisse haben sie geliefert (\zB Logs von Unit Tests, Testprotokolle der Anwender)?
%	\item Wurde die Anwendung offiziell abgenommen?
%\end{itemize}

Für die Abnahme werden ähnlich wie in der Qualitätskontrolle, Anwendertests von tatsächlichen Benutzern, der Anwendung im vollen Umfang durchgeführt.
Die Benutzer sollen wieder typische Interaktionen der Anwendung durchführen, dazu gehören das Nutzen von Key-Events und das prüfen von vorgenommenen Änderungen aus der Qualitätskontrolle.

%Die Anwendertests umfassten eine Reihe von typischen Interaktionen, die Benutzer mit der Anwendung durchführen würden.
%Dazu gehörten unter anderem das Nutzen von Key-Events, um mit der Benutzeroberfläche zu interagieren und das Überprüfen der Anzeige auf dem Fernseher im Eingangsbereich.

Insgesamt verliefen die Anwendertests erfolgreich, und die Anwendung erwies sich als stabil, zuverlässig und benutzerfreundlich.
Es wurden nur wenige kleinere Probleme gefunden, die schnell behoben werden konnten.

Basierend auf den Ergebnissen der Anwendertests und dem positiven Feedback der Benutzer wurde die Anwendung offiziell abgenommen und für den Produktivbetrieb freigegeben.
Sie wurde erfolgreich bereitgestellt und steht nun den Benutzern zur Verfügung.
%\paragraph{Beispiel}
%Ein Auszug eines Unit Tests befindet sich im \Anhang{app:Test}. Dort ist auch der Aufruf des Tests auf der Konsole des Webservers zu sehen.

%\input{Inhalt/Einfuehrungsphase}
% !TEX root = ../Projektdokumentation.tex
\section{Dokumentation}
\label{sec:Dokumentation}

%\begin{itemize}
%	\item Wie wurde die Anwendung für die Benutzer/Administratoren/Entwickler dokumentiert (\zB Benutzerhandbuch, \acs{API}-Dokumentation)?
%	\item Hinweis: Je nach Zielgruppe gelten bestimmte Anforderungen für die Dokumentation (\zB keine IT-Fachbegriffe in einer Anwenderdokumentation verwenden, aber auf jeden Fall in einer Dokumentation für den IT-Bereich).
%\end{itemize}
%
%\paragraph{Beispiel}
%Ein Ausschnitt aus der erstellten Benutzerdokumentation befindet sich im \Anhang{app:BenutzerDoku}.
%Die Entwicklerdokumentation wurde mittels PHPDoc\footnote{Vgl. \cite{phpDoc}} automatisch generiert. Ein beispielhafter Auszug aus der Dokumentation einer Klasse findet sich im \Anhang{app:Doc}.
%\subsection{Projektdokumentation}
%\label{sec:Projektdokumentation}
Die vorliegende Projektdokumentation gibt einen Überblick über die Entstehung und Umsetzung des Projekts.
Sie ist in LaTeX10 verfasst worden, orientiert sich an vorhandenen Projektdokumentationen sowie an den Vorgaben der Prüfungsordnung und den offiziellen Richtlinien der \ac{IHK} zur Projektdokumentation.
% !TEX root = ../Projektdokumentation.tex
\section{Fazit} 
\label{sec:Fazit}

\subsection{Soll-/Ist-Vergleich}
\label{sec:SollIstVergleich}

%\begin{itemize}
%	\item Wurde das Projektziel erreicht und wenn nein, warum nicht?
%	\item Ist der Auftraggeber mit dem Projektergebnis zufrieden und wenn nein, warum nicht?
%	\item Wurde die Projektplanung (Zeit, Kosten, Personal, Sachmittel) eingehalten oder haben sich Abweichungen ergeben und wenn ja, warum?
%	\item Hinweis: Die Projektplanung muss nicht strikt eingehalten werden. Vielmehr sind Abweichungen sogar als normal anzusehen. Sie müssen nur vernünftig begründet werden (\zB durch Änderungen an den Anforderungen, unter-/überschätzter Aufwand).
%\end{itemize}
Das Projektziel, eine innovative Lösung für die Gästebegrüßung und den Informationsfluss zu entwickeln, wurde erreicht.
Die entwickelte Lösung umfasst eine auflistung von Termin-Info-Karten, Live-Video-Streams, einen laufenden Info-Text und einer Uhrzeit- und Datumsangabe.
Diese Elemente tragen effektiv zur Gästebegrüßung und Informationsbereitstellung bei und vermitteln einen positiven ersten Eindruck.
Jedoch gab es einige Herausforderungen während der Implementierung, die zu kleinen Abweichungen führten.


Der Auftraggeber ist zufrieden mit dem Projektergebnis, da die entwickelte Lösung die wesentlichen Anforderungen erfüllt.
Dennoch gibt es Punkte, die in Zukunft noch verbessert werden könnten, wie etwa die Integration zusätzlicher Funktionen und Dienste.

\paragraph{Änderungen der Zeitplanung}
Wie in Tabelle~\ref{tab:Vergleich} zu erkennen ist, wurde die Zeitplanung durch unerwartete Hindernisse geändert.
Als Resultat wurde an anderen Stellen Zeit eingespart.
\tabelle{Soll-/Ist-Vergleich}{tab:Vergleich}{Zeitnachher.tex}


\subsection{Lessons Learned}
\label{sec:LessonsLearned}
%\begin{itemize}
%	\item Was hat der Prüfling bei der Durchführung des Projekts gelernt (\zB Zeitplanung, Vorteile der eingesetzten Frameworks, Änderungen der Anforderungen)?
%\end{itemize}
Durch die sehr selbstständige Projektdurchführung wurde der Autor mit allen Prozessbereichen
der Softwareentwicklung in Kontakt gebracht.


Während der Durchführung des Projekts hat der Autor eine Reihe wichtiger Lektionen gelernt.
Er erkannte die entscheidende Bedeutung einer realistischen Zeitplanung und lernte, wie wichtig es ist, Pufferzeiten für unvorhergesehene Verzögerungen oder technische Herausforderungen einzuplanen.
Die Erfahrung verdeutlichte, dass die Einhaltung des Zeitplans oft schwieriger ist als zunächst angenommen und dass Flexibilität und Anpassungsfähigkeit entscheidend sind.


Des Weiteren erwies sich die Verwendung von \ac{TS} in Verbindung mit dem React-Framework als äußerst vorteilhaft.
Die Typsicherheit von \ac{TS} half dabei, Fehler frühzeitig zu erkennen und die Codequalität zu verbessern, während React eine effiziente Entwicklung von benutzerfreundlichen \ac{UI}-Komponenten ermöglichte.
Diese Erfahrung bestätigte die Vorteile moderner Frameworks für die Entwicklung hochwertiger Anwendungen.

\subsection{Ausblick}
\label{sec:Ausblick}

%\begin{itemize}
%	\item Wie wird sich das Projekt in Zukunft weiterentwickeln (\zB geplante Erweiterungen)?
%\end{itemize}

Für die zukünftige Entwicklung des Projekts sind verschiedene Erweiterungen und Verbesserungen geplant.
Dazu gehören die Erweiterung einer Eingabeoberfläche, der Integration von weiteren Providern und die Anpassung an spezifische Anforderungen.


% Literatur ------------------------------------------------------------------
%\clearpage
%\renewcommand{\refname}{Literaturverzeichnis}
%\bibliography{Bibliographie}
%\bibliographystyle{Allgemein/natdin} % DIN-Stil des Literaturverzeichnisses

%\input{Erklaerung}

% Anhang ---------------------------------------------------------------------
\clearpage
\appendix
\pagenumbering{roman}
% !TEX root = Projektdokumentation.tex
\section{Anhang}
\subsection{Detaillierte Zeitplanung}
\label{app:Zeitplanung}

\tabelleAnhang{ZeitplanungKomplett}

\subsection{Lastenheft (Auszug)}
\label{app:Lastenheft}
Es folgt ein Auszug aus dem Lastenheft mit Fokus auf die Anforderungen:

Die Anwendung muss folgende Anforderungen erfüllen: 
\begin{enumerate}[itemsep=0em,partopsep=0em,parsep=0em,topsep=0em]
\item Authentifizierung
	\begin{enumerate}
	\item Die Anwendung muss in der Lage sein, mit \ac{MS} Azure \ac{AD} zu interagieren, um die Authentifizierung von Benutzern zu ermöglichen.
	\item Die Anwendung soll den \ac{MS} Azure \ac{AD} Authentication Code Flow verwenden, um die Authentifizierung von Benutzern zu ermöglichen. Dies beinhaltet den Austausch von Autorisierungscodes gegen Access-Token und Refresh-Token.
	\item Die Anwendung braucht die nötigen Berechtigungen, um auf den vom Benutzer eingeloggten Kalender über \ac{MS} Azure \ac{AD} und dem zugehörigen Tenant zuzugreifen.
	\end{enumerate}
\item Darstellung der Daten
	\begin{enumerate}
	\item Die Anwendung muss eine Liste aller noch folgenden Termine für den aktuellen Tag anzeigen.
	\item Vergangene Termine müssen 30 Minuten nach Beendigung weiterhin angezeigt werden.
	\item Jeder Besucher soll über die Oberfläche mit folgenden Informationen begrüßt werden.:
	\begin{enumerate}
		\item Dem vollständigen Namen.
		\item Der Firmenzugehörigkeit des Besuchers.
		\item Der Kontaktperson oder Organisator des Termins.
		\item \Ggfs zusätzlicher oder relevanter Informationen zum Termin.
	\end{enumerate}
	\item Die Anwendung muss eine Laufschrift am unteren Bildschirmrand für allgemeine Informationen bereitstellen.
	\item Die Anwendung muss einen für die wartenden Besucher bereitgestellten Live-Stream von \zB den öffentlichen Fernsehprogrammen darstellen.
	\item Die aktuelle Uhrzeit und Datum muss angezeigt werden.
	\end{enumerate}
\item Sonstige Anforderungen
	\begin{enumerate}
	\item Die Anwendung soll als gebaute Executable ausführbar sein.
	\item Die Anwendung soll über den aktuellsten 0 Uhr Kalender-Eintrag konfiguriert werden können. Dazu zählen folgende Konfigurationsmöglichkeiten.:
	\begin{enumerate}
		\item Das setzen des Hintergrundbildes der Anwendung.
		\item Das setzen des Texts für die Laufschrift.
	\end{enumerate}
	\end{enumerate}
\end{enumerate}


\clearpage

\subsection{Verwendete Ressourcen}
\label{app:VerwendeteRessourcen}

\paragraph{Hardware}
\begin{itemize}
	\item MacBook Pro 13 Zoll
	\item Docking Station
	\item 2 Monitore
	\item Bluetooth Headset
	\item Bluetooth Maus und Tastatur
\end{itemize}

\paragraph{Software}
\begin{itemize}
	\item Betriebssystem: macOS Sonoma 14.1.2
	\item Entwicklungsumgebung: \ac{VS} Code mit Erweiterungen
	\item Programmiersprache \ac{TS} 4.5.5
	\item Auszeichnungssprache: \ac{HTML} und \ac{TSX}
	\item Paketmanager Frontend: npm
	\item Diagrammframework: PlantUML\footnote{\ac{UML}} und Draw.io
	\item Stylesheetsprache: \ac{CSS} und Tailwind \ac{CSS}
	\item Versionierung: git/GitLab
	\item Zielplatform: Webbrowser (Chromium)
	\item Meetings: Microsoft Teams und Büro Räumlichkeiten
	\item Unternehmensplatform: Microsoft 365
	\item Mockups: Excalidraw
	\item Dokumentation: LaTeX
	\item Dokumentationseditor: \ac{VS} Code mit Erweiterungen
\end{itemize}

\paragraph{Personal}
\begin{itemize}
	\item Stakeholder für Betreuung, Codereview und Qualitätssicherung, sowohl für Festlegung der Anforderungen an das Projekt und Projektabnahme.
	\item Anwender des Produkts.
\end{itemize}

\subsection{Use Case-Diagramm}
\label{app:UseCase}
\begin{figure}[htb]
\centering
\includegraphicsKeepAspectRatio{use_case.png}{0.7}
\caption{Use Case-Diagramm}
\end{figure}

%\input{Anhang/AnhangPflichtenheft.tex}
%
%\subsection{Datenbankmodell}
%\label{app:Datenbankmodell}
%ER-Modelle kann man auch direkt mit \LaTeX{} zeichnen, siehe \zB \url{http://www.texample.net/tikz/examples/entity-relationship-diagram/}.
%\begin{figure}[htb]
%\centering
%\includegraphicsKeepAspectRatio{database.pdf}{1}
%\caption{Datenbankmodell}
%\end{figure}
\clearpage

\subsection{Oberflächenentwürfe}
\label{app:Entwuerfe}
\begin{figure}[htb]
\centering
\includegraphicsKeepAspectRatio{wireframe-standby.png}{0.7}
\caption{Wireframe der Standby-Ansicht}
\end{figure}

\begin{figure}[htb]
\centering
\includegraphicsKeepAspectRatio{wireframe-fullscreen.png}{0.7}
\caption{Wireframe der Fullscreen-Ansicht}
\end{figure}

\begin{figure}[htb]
\centering
\includegraphicsKeepAspectRatio{wireframe-event.png}{0.7}
\caption{Wireframe der Ansicht mit einem Termin}
\end{figure}

\begin{figure}[htb]
    \centering
    \includegraphicsKeepAspectRatio{wireframe-mehr-events.png}{0.7}
    \caption{Wireframe der Ansicht mit mehreren Terminen}
\end{figure}

\clearpage
\subsection{Produktfotos der Anwendung}
\label{Produktfotos}
\begin{figure}[htb]
\centering
\includegraphicsKeepAspectRatio{Produktfoto_1.jpg}{1}
\caption{Ansicht im Standby}
\end{figure}
\clearpage
\begin{figure}[htb]
    \centering
    \includegraphicsKeepAspectRatio{Produktfoto_2.jpg}{1}
    \caption{Vollbildansicht vom Live-Video}
\end{figure}
\clearpage
\begin{figure}[htb]
    \centering
    \includegraphicsKeepAspectRatio{Produktfoto_3.jpg}{1}
    \caption{Ansicht mit einem Termin}
\end{figure}
\clearpage
\begin{figure}[htb]
    \centering
    \includegraphicsKeepAspectRatio{Produktfoto_4.jpg}{1}
    \caption{Ansicht mit mehreren Terminen}
\end{figure}
\clearpage

%\input{Anhang/AnhangDoc.tex}
%\clearpage
%\input{Anhang/AnhangTest.tex}

\subsection{Main-Komponente Logik-Loop}
\label{app:MAINLOOP}
Main Logic-Loop für das Pulling der Kalendereinträge
\lstinputlisting[language=typescript, caption={Main-Komponente Logik-Loop}]{Listings/main.ts}
\clearpage

\subsection{Main-Komponente IPC}
\label{app:MAINIPC}
Main \ac{IPC} für Kommunikation zwischen Renderer und Main
\lstinputlisting[language=typescript, caption={Main-Komponente IPC}]{Listings/ipcMain.ts}
\clearpage

\subsection{Projektstruktur}
\label{app:Projektstruktur}
\begin{figure}[htb]
\centering
\includegraphicsKeepAspectRatio{Projektstruktur.drawio.png}{1}
\caption{Projektstruktur}
\end{figure}
\clearpage

%\input{Anhang/AnhangBenutzerDoku.tex}


\end{document}
