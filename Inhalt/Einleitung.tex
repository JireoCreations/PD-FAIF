% !TEX root = ../Projektdokumentation.tex
\section{Einleitung}
\label{sec:Einleitung}


\subsection{Projektumfeld} 
\label{sec:Projektumfeld}
%\begin{itemize}
%	\item Kurze Vorstellung des Ausbildungsbetriebs (Geschäftsfeld, Mitarbeiterzahl \usw)
%	\item Wer ist Auftraggeber/Kunde des Projekts?
%\end{itemize}
Seit 2005 gibt es CSTx Software Engineering GmbH und im Jahr 2008 wurde CSTx zum Umsetzungspartner für Volkswagen AG.
Durch den Großkunden konnte sich die CSTx in der Automobilbranche einen festen Platz sichern.
Im Jahr 2018 gab es eine Umstrukturierung der Holding und die Gründung der drei Tochterunternehmen kam zustande.
Insgesamt ist CSTx ein Mittelständiger Betrieb mit einer Mannschaftsgröße von knapp 100 Mitarbeitern.
Die Ausbildung und die Projektarbeit findet in der Tochterfirma CSTx Enterprise Solutions GmbH statt.

%\paragraph{Projektbezug}
%Realisiert wird das Projekt als eigenständige Komplettlösung.
%Auftraggeber und Steakholder ist Mark Barrenscheen, ein COO der CSTx Software Solutions GmbH.


\subsection{Projektziel} 
\label{sec:Projektziel}
%\begin{itemize}
%	\item Worum geht es eigentlich?
%	\item Was soll erreicht werden?
%\end{itemize}
Es soll eine innovative Lösung für die Gästebegrüßung und den Informationsfluss entwickelt werden. 
Dies umfasst die Darstellung einer auf den Kunden angepasste Info-Karte, die Integration von Live-Video-Streams, einem laufenden Info-Text und einer Uhrzeitangabe. 
Durch diese Elemente sollen Gäste effektiv und ansprechend begrüßt werden und gleichzeitig wichtige Informationen erhalten. 
Das Hauptziel besteht darin, ein visuell ansprechendes und informatives Tool zu schaffen, das die Aufmerksamkeit der Besucher auf sich zieht und einen positiven ersten Eindruck vermittelt.


\subsection{Projektbegründung} 
\label{sec:Projektbegruendung}
%\begin{itemize}
%	\item Warum ist das Projekt sinnvoll (\zB Kosten- oder Zeitersparnis, weniger Fehler)?
%	\item Was ist die Motivation hinter dem Projekt?
%\end{itemize}
Es besteht eine Notwendigkeit einer zeitgemäßen und effizienten Gästebegrüßung sowie Informationsbereitstellung.
Das Unternehmen strebt nach einem Produkt, das Prozesse für den Informationsaustausch vereinfacht und beschleunigt.
Zudem soll es äußerst flexibel anpassbar sein und sowohl für Kunden als auch Mitarbeiter einen Mehrwert bieten.
Durch die Digitalisierung der Begrüßungs- und Informationsprozesse soll ein innovatives und professionelles Ambiente geschaffen werden, das die Attraktivität der Einrichtung steigert und die Zufriedenheit der Besucher erhöht.


%Die Implementierung einer digitalen Lösung bietet zahlreiche Vorteile, darunter die Reduzierung von benötigten Räumlichkeiten Für Informationsmaterialien und den dazugehörigen Aufwand.
%Zusätzlich ermöglicht sie eine schnellere Aktualisierung von Informationen, was zu einer zeitlichen Ersparnis führt und einen nahtlosen übergang von updates sicherstellt.
%Die Motivation hinter dem Projekt ist die Schaffung eines modernen und benutzerfreundlichen Systems, das Gäste anspricht und gleichzeitig die Arbeitsabläufe optimiert.
%Durch die Digitalisierung der Begrüßungs- und Informationsprozesse soll ein innovatives und professionelles Ambiente geschaffen werden, das die Attraktivität der Einrichtung steigert und die Zufriedenheit der Besucher erhöht.


\subsection{Projektschnittstellen} 
\label{sec:Projektschnittstellen}
%\begin{itemize}
%	\item Mit welchen anderen Systemen interagiert die Anwendung (technische Schnittstellen)?
%	\item Wer genehmigt das Projekt \bzw stellt Mittel zur Verfügung?
%	\item Wer sind die Benutzer der Anwendung?
%	\item Wem muss das Ergebnis präsentiert werden?
%\end{itemize}
Die Anwendung interagiert mit Microsoft Azure AD für die Authentifizierung und den Zugriff auf Kalendereinträge.
Die Genehmigung des Projekts sowie die Bereitstellung von Ressourcen erfolgt durch den Stakeholder und Entscheidungsträger, wie beispielsweise die die Geschäftsführung.
Jeder Mitarbeiter, der über einen Azure-Account der Organisation verfügt, kann sich in die Anwendung einloggen und diese nutzen.
Die Anwendung zeigt die kommenden Kalendereinträge des Tages an und kann somit auch als Dashboard für den kommenden Arbeitstag einzelner Mitarbeiter verwendet werden.
Das Ergebnis des Projekts muss Mark Barrenscheen präsentiert werden, der das Projekt initiiert hat und sowohl Stakeholder als auch Hauptanwender ist, der die Anwendung vorrangig nutzt.

\subsection{Projektabgrenzung} 
\label{sec:Projektabgrenzung}
%\begin{itemize}
%	\item Was ist explizit nicht Teil des Projekts (\insb bei Teilprojekten)?
%\end{itemize}
Die Einrichtung der Authentifizierung seitens Azure AD sowie das Deployment sind nicht Teil des Projekts.
Darüber hinaus wird ausdrücklich gewünscht, dass keine Datenbank verwendet wird, da lediglich das Auslesen der Kalendereinträge erforderlich ist.
Die Einbindung weiterer Integrationen von Authentifizierungs- und Kalendersupport von \zB Google, ist nicht Teil des Projekts.