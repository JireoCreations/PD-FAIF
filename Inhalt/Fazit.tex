% !TEX root = ../Projektdokumentation.tex
\section{Fazit} 
\label{sec:Fazit}

\subsection{Soll-/Ist-Vergleich}
\label{sec:SollIstVergleich}

%\begin{itemize}
%	\item Wurde das Projektziel erreicht und wenn nein, warum nicht?
%	\item Ist der Auftraggeber mit dem Projektergebnis zufrieden und wenn nein, warum nicht?
%	\item Wurde die Projektplanung (Zeit, Kosten, Personal, Sachmittel) eingehalten oder haben sich Abweichungen ergeben und wenn ja, warum?
%	\item Hinweis: Die Projektplanung muss nicht strikt eingehalten werden. Vielmehr sind Abweichungen sogar als normal anzusehen. Sie müssen nur vernünftig begründet werden (\zB durch Änderungen an den Anforderungen, unter-/überschätzter Aufwand).
%\end{itemize}
Das Projektziel, eine innovative Lösung für die Gästebegrüßung und den Informationsfluss zu entwickeln, wurde erreicht.
Die entwickelte Lösung umfasst eine Auflistung von Termin-Info-Karten, Live-Video-Streams, einen laufenden Info-Text und einer Uhrzeit- und Datumsangabe.
Diese Elemente tragen effektiv zur Gästebegrüßung und Informationsbereitstellung bei und vermitteln einen positiven ersten Eindruck.
Jedoch gab es einige Herausforderungen während der Implementierung, die zu kleinen Abweichungen führten.


Der Auftraggeber ist zufrieden mit dem Projektergebnis, da die entwickelte Lösung die wesentlichen Anforderungen erfüllt.
Dennoch gibt es Punkte, die in Zukunft noch verbessert werden könnten, wie etwa die Integration zusätzlicher Funktionen und Dienste.

\paragraph{Änderungen der Zeitplanung}
Wie in Tabelle~\ref{tab:Vergleich} zu erkennen ist, wurde die Zeitplanung durch unerwartete Hindernisse geändert.
Als Resultat wurde an anderen Stellen Zeit eingespart.
\tabelle{Soll-/Ist-Vergleich}{tab:Vergleich}{Zeitnachher.tex}


\subsection{Lessons Learned}
\label{sec:LessonsLearned}
%\begin{itemize}
%	\item Was hat der Prüfling bei der Durchführung des Projekts gelernt (\zB Zeitplanung, Vorteile der eingesetzten Frameworks, Änderungen der Anforderungen)?
%\end{itemize}
Durch die sehr selbstständige Projektdurchführung wurde der Autor mit allen Prozessbereichen
der Softwareentwicklung in Kontakt gebracht.


Während der Durchführung des Projekts hat der Autor eine Reihe wichtiger Lektionen gelernt.
Er erkannte die entscheidende Bedeutung einer realistischen Zeitplanung und lernte, wie wichtig es ist, Pufferzeiten für unvorhergesehene Verzögerungen oder technische Herausforderungen einzuplanen.
Die Erfahrung verdeutlichte, dass die Einhaltung des Zeitplans oft schwieriger ist als zunächst angenommen und dass Flexibilität und Anpassungsfähigkeit entscheidend sind.


Des Weiteren erwies sich die Verwendung von \ac{TS} in Verbindung mit dem React-Framework als äußerst vorteilhaft.
Die Typsicherheit von \ac{TS} half dabei, Fehler frühzeitig zu erkennen und die Codequalität zu verbessern, während React eine effiziente Entwicklung von benutzerfreundlichen \ac{UI}-Komponenten ermöglichte.
Diese Erfahrung bestätigte die Vorteile moderner Frameworks für die Entwicklung hochwertiger Anwendungen.

\subsection{Ausblick}
\label{sec:Ausblick}

%\begin{itemize}
%	\item Wie wird sich das Projekt in Zukunft weiterentwickeln (\zB geplante Erweiterungen)?
%\end{itemize}

Für die zukünftige Entwicklung des Projekts sind verschiedene Erweiterungen und Verbesserungen geplant.
Dazu gehören die Erweiterung einer Eingabeoberfläche, der Integration von weiteren Providern und die Anpassung an spezifische Anforderungen.
