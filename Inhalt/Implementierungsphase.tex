% !TEX root = ../Projektdokumentation.tex
\section{Implementierungsphase} 
\label{sec:Implementierungsphase}

%\subsection{Implementierung der Datenstrukturen}
%\label{sec:ImplementierungDatenstrukturen}
%
%\begin{itemize}
%	\item Beschreibung der angelegten Datenbank (\zB Generierung von \acs{SQL} aus Modellierungswerkzeug oder händisches Anlegen), \acs{XML}-Schemas \usw.
%\end{itemize}


\subsection{Implementierung der Benutzeroberfläche}
\label{sec:ImplementierungBenutzeroberflaeche}

%\begin{itemize}
%	\item Beschreibung der Implementierung der Benutzeroberfläche, falls dies separat zur Implementierung der Geschäftslogik erfolgt (\zB bei \acs{HTML}-Oberflächen und Stylesheets).
%	\item \Ggfs Beschreibung des Corporate Designs und dessen Umsetzung in der Anwendung.
%	\item Screenshots der Anwendung
%\end{itemize}
Die Implementierung der Benutzeroberfläche erfolgt separat von der Implementierung der Geschäftslogik.
In diesem Fall wird das React-Framework verwendet, um die verschiedenen \ac{UI}-Komponenten zu entwickeln und zu gestalten.

Die Benutzeroberfläche wird mithilfe von \ac{TSX} erstellt, einer Kombination aus \ac{TS} und \ac{JSX}, welches einer Syntaxerweiterung von \ac{JS} ist, die es ermöglicht, \ac{HTML}-ähnliche Elemente direkt in \ac{JS}-Code einzubetten.
Dadurch können die verschiedenen \ac{UI}-Komponenten in einer klaren und strukturierten Weise erstellt werden, wobei die Prozesse zur formatierung und das Markup eng miteinander verbunden sind.

Für das Styling wird Tailwind \ac{CSS}, ein Utility-First \ac{CSS}-Framework, das eine Reihe von vorgefertigten Utility-Klassen bereitstellt, genutzt.
Dadurch wird die Entwicklung von \ac{CSS} stark vereinfacht und beschleunigt, da keine separaten Stylesheets erstellt werden müssen.

\paragraph{Produkt in Aktion}
Bilder vom Produkt im Eingangsbereich mit Dummy-Daten befinden sich im \Anhang{Produktfotos}.


\subsection{Implementierung der Geschäftslogik}
\label{sec:ImplementierungGeschaeftslogik}
%\begin{itemize}
%	\item Beschreibung des Vorgehens bei der Umsetzung/Programmierung der entworfenen Anwendung.
%	\item \Ggfs interessante Funktionen/Algorithmen im Detail vorstellen, verwendete Entwurfsmuster zeigen.
%	\item Quelltextbeispiele zeigen.
%	\item Hinweis: Wie in Kapitel~\ref{sec:Einleitung}: \nameref{sec:Einleitung} zitiert, wird nicht ein lauffähiges Programm bewertet, sondern die Projektdurchführung. Dennoch würde ich immer Quelltextausschnitte zeigen, da sonst Zweifel an der tatsächlichen Leistung des Prüflings aufkommen können.
%\end{itemize}
Die Geschäftslogik umfasst mehrere Schlüsselfunktionen wie das Bereitstellen von Kalenderdaten wofür die Kommunikation mit \ac{MS} Azure \ac{AD} benötigt wird.
Zusätzlich wird die Verarbeitung von Benutzerinteraktionen und die Verwaltung von Zuständen bestandteil der Geschäftslogik.


Für die Bereitstellung von Kalenderdaten ist es wichtig, Daten aus \ac{MS} Azure \ac{AD} abzurufen, sie zu parsen und entsprechend vorzubereiten, um kommende Termine und Veranstaltungen im Renderer anzuzeigen.
Diese Daten müssen dynamisch in die Benutzeroberfläche integriert werden, um den Benutzern einen aktuellen Überblick zu bieten, dies wird erreicht, indem eine \ac{IPC} basierend auf dem Observer-Pattern erstellt wird.


Die Verwaltung von Benutzerinteraktionen ist ein weiterer zentraler Aspekt der Geschäftslogik.
Da die Anwendung von einer aktiven Authentifikation abhängig ist, müssen Funktionalitäten definiert und entsprechende Aktionen zugewiesen werden, die den Authorization Code Flow von \ac{MS} Azure \ac{AD} nutzen.
Dies erfordert das Implementieren von \ac{MSAL} und einer korrekten handhabung von Access-Token.


Die Verwaltung von Zuständen ist ein entscheidender Aspekt, um sicherzustellen, dass die Benutzeroberfläche der Anwendung korrekt aktualisiert wird und auf Änderungen reagiert.
Dies umfasst, das Definieren und Aktualisieren von Konfigurationen und der zu nutzenden Session, sowie das Aktualisieren vom aktuellen Zustand im Renderer.


\paragraph{Teile der Implementierung}
Der Logik-Loop für das Pulling der Kalendereinträge in der \texttt{Main\-Kom\-po\-nen\-te} findet sich im \Anhang{app:MAINLOOP}.
