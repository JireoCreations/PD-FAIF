% !TEX root = ../Projektdokumentation.tex
\section{Abnahmephase} 
\label{sec:Abnahmephase}

%\begin{itemize}
%	\item Welche Tests (\zB Unit-, Integrations-, Systemtests) wurden durchgeführt und welche Ergebnisse haben sie geliefert (\zB Logs von Unit Tests, Testprotokolle der Anwender)?
%	\item Wurde die Anwendung offiziell abgenommen?
%\end{itemize}

Für die Abnahme werden ähnlich wie in der Qualitätskontrolle, Anwendertests von tatsächliche Benutzern, der Anwendung im vollen Umfang durchgeführt.
Die Benutzer sollen wieder typische Interaktionen der Anwendung durchführen, dazu gehören das Nutzen von Key-Events und vorgenommene Änderung aus der Qualitätskontrolle.

%Die Anwendertests umfassten eine Reihe von typischen Interaktionen, die Benutzer mit der Anwendung durchführen würden.
%Dazu gehörten unter anderem das Nutzen von Key-Events, um mit der Benutzeroberfläche zu interagieren und das Überprüfen der Anzeige auf dem Fernseher im Eingangsbereich.

Insgesamt verliefen die Anwendertests erfolgreich, und die Anwendung erwies sich als stabil, zuverlässig und benutzerfreundlich.
Es wurden nur wenige kleinere Probleme gefunden, die schnell behoben werden konnten.

Basierend auf den Ergebnissen der Anwendertests und dem positiven Feedback der Benutzer wurde die Anwendung offiziell abgenommen und für den Produktivbetrieb freigegeben.
Sie wurde erfolgreich bereitgestellt und steht nun den Benutzern zur Verfügung.
%\paragraph{Beispiel}
%Ein Auszug eines Unit Tests befindet sich im \Anhang{app:Test}. Dort ist auch der Aufruf des Tests auf der Konsole des Webservers zu sehen.
