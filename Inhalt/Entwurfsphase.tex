% !TEX root = ../Projektdokumentation.tex
\section{Vorbereitungsphase}
\label{sec:Vorbereitungsphase}

\subsection{Zielplattform}
\label{sec:Zielplattform}
%\begin{itemize}
%	\item Beschreibung der Kriterien zur Auswahl der Zielplattform (\ua Programmiersprache, Datenbank, Client/Server, Hardware).
%\end{itemize}
In Bezug auf die Programmiersprache und das Framework fiel die Wahl auf \ac{TS} in Verbindung mit dem React-Framework.
\ac{TS} bietet die Möglichkeit, eine robuste und wartbare Codebasis zu entwickeln, während React eine interaktive Benutzeroberfläche ermöglicht.
Diese Kombination erleichtert die Entwicklung einer benutzerfreundlichen Anwendung und bietet die Möglichkeit zur Wiederverwendung von Code.


Die Entscheidung für die Laufzeitumgebung fiel auf Electron, um eine plattformübergreifende Kompatibilität sicherzustellen und die Nutzung von Webtechnologien zu ermöglichen.
Durch Electron kann die Anwendung als Desktop-App ausgeführt werden und ist somit auf verschiedenen Betriebssystemen wie Windows, macOS und Linux nutzbar.


\subsection{Architekturdesign}
\label{sec:Architekturdesign}
%\begin{itemize}
%	\item Beschreibung und Begründung der gewählten Anwendungsarchitektur (\zB \acs{MVC}).
%	\item \Ggfs Bewertung und Auswahl von verwendeten Frameworks sowie \ggfs eine kurze Einführung in die Funktionsweise des verwendeten Frameworks.
%\end{itemize}
Die gewählte Architektur basiert auf dem Konzept des \ac{CP}-Musters, das eine klare Trennung zwischen der Logik der Anwendung und der Darstellung der Benutzeroberfläche ermöglicht.
Dieses Muster wird oft in Kombination mit dem React-Framework verwendet und ermöglicht eine effiziente Entwicklung und Wartung der Anwendung


Im Rahmen dieser Architektur werden die verschiedenen Komponenten der Anwendung in Container-Komponenten und Präsentationskomponenten aufgeteilt.
Die Container-Komponenten sind für die Logik der Anwendung zuständig, während die Präsentationskomponenten die Darstellung der Benutzeroberfläche übernehmen.
Dadurch wird eine klare Trennung von Daten und Darstellung erreicht, was die Wiederverwendbarkeit und Testbarkeit der Komponenten verbessert.


Die Wahl des React-Frameworks unterstützt diese Architektur, da React die Entwicklung von wiederverwendbaren \ac{UI}-Komponenten erleichtert und eine effiziente Aktualisierung der Benutzeroberfläche ermöglicht.
React verwendet eine virtuelle \ac{DOM}, um Änderungen effizient zu verarbeiten und die Benutzeroberfläche schnell zu aktualisieren.

%\paragraph{Beispiel}
%Anhand der Entscheidungsmatrix in Tabelle~\ref{tab:Entscheidungsmatrix} wurde für die Implementierung der Anwendung das \acs{PHP}-Framework Symfony\footnote{\Vgl \citet{Symfony}.} ausgewählt.
%
%\tabelle{Entscheidungsmatrix}{tab:Entscheidungsmatrix}{Nutzwert.tex}


\subsection{Entwurf der Benutzeroberfläche}
\label{sec:Benutzeroberflaeche} 
%\begin{itemize}
%	\item Entscheidung für die gewählte Benutzeroberfläche (\zB GUI, Webinterface).
%	\item Beschreibung des visuellen Entwurfs der konkreten Oberfläche (\zB Mockups, Menüführung).
%	\item \Ggfs Erläuterung von angewendeten Richtlinien zur Usability und Verweis auf Corporate Design.
%\end{itemize}
Die Benutzeroberfläche besteht aus vier Hauptkomponenten, die auf einen Blick alle relevanten Informationen anzeigen: die Uhrzeit und das Datum, den Live-Video-Feed, die Laufschrift und die Terminauflistung.
Die Anordnung der Komponenten erfolgt so, dass sie übersichtlich und leicht erkennbar sind.

Die Uhrzeit und das Datum werden prominent oben rechts oder oben links auf der Anzeige platziert, um sie leicht sichtbar zu machen.
Dies ermöglicht den Besuchern, jederzeit die aktuelle Zeit und das Datum im Blick zu haben.

Der Live-Video-Feed nimmt den Bereich oben links oder unten links der Benutzeroberfläche ein.
Dadurch können sie wichtige Ereignisse oder Informationen in Echtzeit verfolgen.

Die Laufschrift wird am unteren Rand der Anzeige positioniert und zeigt kontinuierlich wichtige Nachrichten oder Mitteilungen von oder über das Unternehmen an.
Dies ermöglicht es den Besuchern, relevante Informationen schnell zu erfassen, selbst wenn sie sich nicht aktiv auf die Anzeige konzentrieren.

Die Terminauflistung wird angezeigt, wenn Termine vorhanden sind, und zeigt eine Liste der kommenden Ereignisse oder Besprechungen über die gesamte rechte Seite an.
Wenn keine Termine vorhanden sind, wird die Terminauflistung ausgeblendet, um Platz für die anderen Komponenten zu schaffen.
Termine innerhalb der Auflistung werden genutzt um letzlich den Besucher zu begrüßen.

Es gibt kein traditionelles Menü oder \ac{UI} in der Anwendung.
Die Anwendung wird ausschließlich von Hotkeys gesteuert, die von den Benutzern verwendet werden können, um zwischen den verschiedenen Ansichten zu navigieren oder Aktionen auszuführen.
Darüber hinaus reagiert die Anwendung auf Kalenderupdates, um Änderungen in den Terminen sofort zu reflektieren und die Anzeige entsprechend anzupassen.

\paragraph{Wireframes}
Entwürfe für die Oberfläche finden sich im \Anhang{app:Entwuerfe}.


%\subsection{Datenmodell}
%\label{sec:Datenmodell}
%
%\begin{itemize}
%	\item Entwurf/Beschreibung der Datenstrukturen (\zB \acs{ERM} und/oder Tabellenmodell, \acs{XML}-Schemas) mit kurzer Beschreibung der wichtigsten (!) verwendeten Entitäten.
%\end{itemize}
%
%\paragraph{Beispiel}
%In \Abbildung{ER} wird ein \ac{ERM} dargestellt, welches lediglich Entitäten, Relationen und die dazugehörigen Kardinalitäten enthält.
%
%\begin{figure}[htb]
%\centering
%\includegraphicsKeepAspectRatio{ERDiagramm.pdf}{0.6}
%\caption{Vereinfachtes ER-Modell}
%\label{fig:ER}
%\end{figure}


\subsection{Geschäftslogik}
\label{sec:Geschaeftslogik}
%\begin{itemize}
%	\item Modellierung und Beschreibung der wichtigsten (!) Bereiche der Geschäftslogik (\zB mit Kom\-po\-nen\-ten-, Klassen-, Sequenz-, Datenflussdiagramm, Programmablaufplan, Struktogramm, \ac{EPK}).
%	\item Wie wird die erstellte Anwendung in den Arbeitsfluss des Unternehmens integriert?
%\end{itemize}

Eine konkrete Geschäftslogik ist nicht Teil des Projekts und wird von \ac{MS} Azure \ac{AD} genutzt.
Es wird lediglich das benötigte Anfordern und die dazugehörige Darstellung von Kontakt und Termindaten implementiert.

Dieses Projekts ist klar strukturiert und ist unterteilt zwischen einer Renderer- und Main-Komponente.
Hier wird das Electron-Framework verwendet, um sowohl eine Chromium Instanz als auch die Kommunikation zwischen den beiden Komponente mittels des Observer-Patterns bereitzustellen.
Der Renderer nutzt React, \ac{TS} und Tailwind \ac{CSS}, um eine interaktive und ansprechende Oberfläche zu gestalten und integriert zusätzlich einen Key-Handler für eine vereinfachte Benutzerinteraktion.
Innerhalb des Renderers sind verschiedene visuelle Komponenten wie Uhrzeit und Datum, Live-Video, Laufschrift, eine auflistung von Terminen und Login-Funktion integriert.

Die Main-Komponente ist für die Verwaltung von Fenstern, Stores und Utilities verantwortlich und nutzt die Integration von Helper-Funktionen.
Die Authentifizierung erfolgt über \ac{MSAL}, während eine \ac{MS} Graph-Schnittstelle über Axios implementiert wird, um auf die Kalendereinträge von dem Benutzer zuzugreifen.

\paragraph{Diagramm}
Eine vereinfachte Darstellung der Logik ist im \Anhang{app:Projektstruktur} zu finden.

%\Abbildung{Modulimport} zeigt den grundsätzlichen Programmablauf beim Einlesen eines Moduls als \ac{EPK}.
%\begin{figure}[htb]
%\centering
%\includegraphicsKeepAspectRatio{modulimport.pdf}{0.9}
%\caption{Prozess des Einlesens eines Moduls}
%\label{fig:Modulimport}
%\end{figure}


\subsection{Maßnahmen zur Qualitätssicherung}
\label{sec:Qualitaetssicherung}
%\begin{itemize}
%	\item Welche Maßnahmen werden ergriffen, um die Qualität des Projektergebnisses (siehe Kapitel~\ref{sec:Qualitaetsanforderungen}: \nameref{sec:Qualitaetsanforderungen}) zu sichern (\zB automatische Tests, Anwendertests)?
%	\item \Ggfs Definition von Testfällen und deren Durchführung (durch Programme/Benutzer).
%\end{itemize}
Um die Qualität des Projektergebnisses sicherzustellen, werden ausschließlich Anwendertests durchgeführt.
Diese Tests werden von dem Entwickler und tatsächlichen Benutzern der Anwendung durchgeführt, um sicherzustellen, dass sie den Anforderungen und Erwartungen des Stakeholders entsprechen.

Die Anwendertests umfassen eine Reihe von Szenarien und Aufgaben, die die Benutzer durchführen sollen.
Diese können \bspw das Live-Video Signal durch die Key-Events wechseln oder einen 0 Uhr Kalendereintrag setzen, um die Konfiguration zu testen.

Es liegt der Fokus darauf, die Anwendung in einer realen Umgebung zu testen und Feedback von den Benutzern einzuholen.
Dies ermöglicht es, potenzielle Probleme oder Verbesserungsmöglichkeiten frühzeitig zu identifizieren und direkt in die Entwicklung einzubeziehen.

Die Anwendertests werden kontinuierlich während des Entwicklungsprozesses durchgeführt, somit wird die Anforderungen und die Qualität des Projektergebnisses gewährleistet.


%\subsection{Pflichtenheft/Datenverarbeitungskonzept}
%\label{sec:Pflichtenheft}
%\begin{itemize}
%	\item Auszüge aus dem Pflichtenheft/Datenverarbeitungskonzept, wenn es im Rahmen des Projekts erstellt wurde.
%\end{itemize}
%
%\paragraph{Beispiel}
%Ein Beispiel für das auf dem Lastenheft (siehe Kapitel~\ref{sec:Lastenheft}: \nameref{sec:Lastenheft}) aufbauende Pflichtenheft ist im \Anhang{app:Pflichtenheft} zu finden.
