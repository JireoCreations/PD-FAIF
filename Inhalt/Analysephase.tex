% !TEX root = ../Projektdokumentation.tex
\section{Analysephase} 
\label{sec:Analysephase}


\subsection{Ist-Analyse} 
\label{sec:IstAnalyse}
%\begin{itemize}
%	\item Wie ist die bisherige Situation (\zB bestehende Programme, Wünsche der Mitarbeiter)?
%	\item Was gilt es zu erstellen/verbessern?
%\end{itemize}
Bisher müssen Kunden auf ihren Ansprechpartner im Flur warten und das Unternehmen wünscht sich eine digitale Lösung, diese automatisiert zu begrüßen und vorab über ihren Termin zu informieren.


\subsection{Wirtschaftlichkeitsanalyse}
\label{sec:Wirtschaftlichkeitsanalyse}
%\begin{itemize}
%	\item Lohnt sich das Projekt für das Unternehmen?
%\end{itemize}
Dieses Projekt bietet eine Vielzahl von Vorteile.
Durch die Automatisierung der Gästebegrüßung und Informationsbereitstellung wird die Effizienz im Eingangsbereich erheblich gesteigert.
%Dies optimiert nicht nur die Arbeitsabläufe, sondern hinterlässt auch einen positiven ersten Eindruck bei den Gästen und könnte möglicherweise ihre Zufriedenheit steigern.
%Dadurch könnte eine stärkere Kundenbindung erreicht werden.
Basierend auf den potenziellen Effizienzsteigerungen im Eingangsbereich sowie der Möglichkeit, einen positiven Eindruck bei den Gästen zu hinterlassen und deren Zufriedenheit zu steigern, scheint das Projekt eine lohnende Investition zu sein.


\subsubsection{\gqq{Make or Buy}-Entscheidung}
\label{sec:MakeOrBuyEntscheidung}
%\begin{itemize}
%	\item Gibt es vielleicht schon ein fertiges Produkt, dass alle Anforderungen des Projekts abdeckt?
%	\item Wenn ja, wieso wird das Projekt trotzdem umgesetzt?
%\end{itemize}
Das Unternehmen entscheidet sich für die interne Entwicklung, da es sicherstellen will, dass das Design genau seinen Vorstellungen entspricht und die Informationen präzise auf seine Bedürfnisse zugeschnitten sind.
Durch die interne Entwicklung behält das Unternehmen die volle Kontrolle über den Prozess, was die geforderte Flexibilität ermöglicht, um auf Änderungen zu reagieren.


\subsubsection{Projektkosten}
\label{sec:Projektkosten}
%\begin{itemize}
%	\item Welche Kosten fallen bei der Umsetzung des Projekts im Detail an (\zB Entwicklung, Einführung/Schulung, Wartung)?
%\end{itemize}

\paragraph{Rechnung (verkürzt)}
Die Kosten für die Durchführung des Projekts setzen sich sowohl aus Personal-, als auch aus Ressourcenkosten zusammen.
Laut Tarifvertrag verdient ein Auszubildender im dritten Lehrjahr pro Monat \eur{900} Brutto.

\begin{eqnarray}
8 \mbox{ h/Tag} \cdot 220 \mbox{ Tage/Jahr} = 1.760 \mbox{ h/Jahr}\\
\eur{900}\mbox{/Monat} \cdot 13,3 \mbox{ Monate/Jahr} = \eur{11.970} \mbox{/Jahr}\\
\frac{\eur{11.970} \mbox{/Jahr}}{1760 \mbox{ h/Jahr}} \approx \eur{6,81}\mbox{/h}
\end{eqnarray}

Es ergibt sich also ein Stundenlohn von \eur{6,81}.
Die Durchführungszeit des Projekts beträgt 80 Stunden. Für die Nutzung von Ressourcen\footnote{Räumlichkeiten, Arbeitsplatzrechner etc.} wird
ein pauschaler Stundensatz von \eur{14} angenommen. Für die anderen Mitarbeiter wird pauschal ein Stundenlohn von \eur{25} angenommen.
Eine Aufstellung der Kosten befindet sich in Tabelle~\ref{tab:Kostenaufstellung} und sie betragen insgesamt \eur{1.859,80}.
\tabelle{Kostenaufstellung}{tab:Kostenaufstellung}{Kostenaufstellung.tex}


\subsubsection{Amortisationsdauer}
\label{sec:Amortisationsdauer}
%\begin{itemize}
%	\item Welche monetären Vorteile bietet das Projekt (\zB Einsparung von Lizenzkosten, Arbeitszeitersparnis, bessere Usability, Korrektheit)?
%	\item Wann hat sich das Projekt amortisiert?
%\end{itemize}
Das Produkt führt voraussichtlich zu einer Verkürzung der Vorbereitungsdauer eines Termins um etwa 20 Minuten.
Für zusätzliche Zeitersparnisse durch eine optimierte Informationsverteilung werden pauschal 30 Minuten einkalkuliert.
Angenommen wird, dass durchschnittlich 10 Kunden zu Besuch kommen und sich 3 Mal pro Woche allgemeine Informationen ändern, während 1 Mitarbeiter dafür verantwortlich ist.
Unter Berücksichtigung dieser Annahmen ergibt sich folgende Berechnung.

\paragraph{Rechnung (verkürzt)}
Bei einem Zeitersparnis von 20 Minuten pro Termin und 30 Minuten für jede Informationsverteilung, bei einem Benutzer und 220 Arbeitstagen im Jahr, wobei sich 10 Termine und 3 Informationsverteilungen pro Woche ereignen, ergibt sich folgendes Gesamtzeitersparnis.
\begin{eqnarray}
220 \mbox{ Tage/Jahr} \cdot (20 \mbox{ min/Termin} \cdot 10 \mbox{ mal/Woche}) = 8800 \mbox{ min/Jahr} \approx 147 \mbox{ h/Jahr}\\
220 \mbox{ Tage/Jahr} \cdot (30 \mbox{ min/IV} \cdot 3 \mbox{ mal/Woche}) = 3960 \mbox{ min/Jahr} \approx 66 \mbox{ h/Jahr}
\end{eqnarray}

Dadurch ergibt sich eine jährliche Einsparung von 
\begin{eqnarray}
(147 \mbox{ h} + 66 \mbox{ h}) \cdot \eur{(25 + 14)}{\mbox{/h}} = \eur{8307}
\end{eqnarray}

Die Amortisationszeit beträgt also $\frac{\eur{1.859,80}}{\eur{8307}\mbox{/Jahr}} \approx 0,02 \mbox{ Jahre} \approx 1 \mbox{ Woche}$.


\subsection{Nutzwertanalyse}
\label{sec:Nutzwertanalyse}
\begin{itemize}
	\item Darstellung des nicht-monetären Nutzens (\zB Vorher-/Nachher-Vergleich anhand eines Wirtschaftlichkeitskoeffizienten). 
\end{itemize}

\paragraph{Beispiel}
Ein Beispiel für eine Entscheidungsmatrix findet sich in Kapitel~\ref{sec:Architekturdesign}: \nameref{sec:Architekturdesign}.


\subsection{Anwendungsfälle}
\label{sec:Anwendungsfaelle}
\begin{itemize}
	\item Welche Anwendungsfälle soll das Projekt abdecken?
	\item Einer oder mehrere interessante (!) Anwendungsfälle könnten exemplarisch durch ein Aktivitätsdiagramm oder eine \ac{EPK} detailliert beschrieben werden. 
\end{itemize}

\paragraph{Beispiel}
Ein Beispiel für ein Use Case-Diagramm findet sich im \Anhang{app:UseCase}.


\subsection{Qualitätsanforderungen}
\label{sec:Qualitaetsanforderungen}
\begin{itemize}
	\item Welche Qualitätsanforderungen werden an die Anwendung gestellt (\zB hinsichtlich Performance, Usability, Effizienz \etc (siehe \citet{ISO9126}))?
\end{itemize}


\subsection{Lastenheft/Fachkonzept}
\label{sec:Lastenheft}
\begin{itemize}
	\item Auszüge aus dem Lastenheft/Fachkonzept, wenn es im Rahmen des Projekts erstellt wurde.
	\item Mögliche Inhalte: Funktionen des Programms (Muss/Soll/Wunsch), User Stories, Benutzerrollen
\end{itemize}

\paragraph{Beispiel}
Ein Beispiel für ein Lastenheft findet sich im \Anhang{app:Lastenheft}. 
