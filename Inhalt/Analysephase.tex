% !TEX root = ../Projektdokumentation.tex
\section{Analysephase} 
\label{sec:Analysephase}


\subsection{Ist-Analyse} 
\label{sec:IstAnalyse}
%\begin{itemize}
%	\item Wie ist die bisherige Situation (\zB bestehende Programme, Wünsche der Mitarbeiter)?
%	\item Was gilt es zu erstellen/verbessern?
%\end{itemize}
%Bisher müssen Kunden auf ihren Ansprechpartner im Flur warten und das Unternehmen wünscht sich
%eine digitale Lösung, diese automatisiert zu begrüßen und vorab über ihren Termin zu informieren.
Bisher müssen Kunden auf ihren Ansprechpartner im Flur warten, was zu einem Verlust an Zeit führt, die für die Aufklärung über den bevorstehenden Termin verwendet werden könnte.
In Anbetracht dessen strebt \ac{CSTx ES} eine digitale Lösung an, um diesen Prozess zu optimieren.
Durch die Einführung eines automatisierten Systems, das Kunden vorab über ihre Termine informiert und sie bei ihrer Ankunft begrüßt, kann die Wartezeit dazu genutzt werden sich über den Termin zu informieren und die Kundenzufriedenheit steigern.

\subsection{Wirtschaftlichkeitsanalyse}
\label{sec:Wirtschaftlichkeitsanalyse}
%\begin{itemize}
%	\item Lohnt sich das Projekt für das Unternehmen?
%\end{itemize}
%Dieses Projekt bietet eine Vielzahl von Vorteilen. Durch die Automatisierung der Gästebegrüßung
%und Informationsbereitstellung wird die Effizienz im Eingangsbereich erheblich gesteigert. Basierend
%auf den potenziellen Effizienzsteigerungen im Eingangsbereich sowie der Möglichkeit, einen positiven
%Eindruck bei den Gästen zu hinterlassen und deren Zufriedenheit zu steigern, scheint das Projekt eine
%lohnende Investition zu sein.
Dieses Projekt bietet eine Vielzahl von Vorteilen.
Durch die Automatisierung der Gästebegrüßung und \ac{IV} wird die Effizienz im Eingangsbereich erheblich gesteigert.

Durch die Implementierung einer digitalen Lösung, die Kunden vorab über ihre Termine informiert und sie bei ihrer Ankunft automatisiert begrüßt, können Zeitverluste minimiert und der gesamte Terminablauf optimiert werden.
Diese Effizienzsteigerungen tragen nicht nur dazu bei, die interne Betriebsabläufe zu verbessern, sondern ermöglichen es auch den Kunden, ihre Zeit effektiver zu nutzen und ihre Erwartungen an einen reibungslosen und professionellen Service zu erfüllen.

Darüber hinaus bietet das Projekt die Möglichkeit, einen bleibenden positiven Eindruck bei den Gästen zu hinterlassen.
Ein reibungsloser Empfangsprozess und eine persönliche Begrüßung vermitteln den Kunden das Gefühl, willkommen und geschätzt zu sein.
Dies stärkt nicht nur die Kundenbindung, sondern trägt auch dazu bei, das Unternehmensimage zu verbessern und neue Geschäftsmöglichkeiten zu schaffen.


\subsubsection{\gqq{Make or Buy}-Entscheidung}
\label{sec:MakeOrBuyEntscheidung}
%\begin{itemize}
%	\item Gibt es vielleicht schon ein fertiges Produkt, dass alle Anforderungen des Projekts abdeckt?
%	\item Wenn ja, wieso wird das Projekt trotzdem umgesetzt?
%\end{itemize}
\ac{CSTx ES} entscheidet sich für die interne Entwicklung, da es sicherstellen will, dass das Design genau seinen Vorstellungen entspricht und die Informationen präzise auf seine Bedürfnisse zugeschnitten sind.
Durch die interne Entwicklung behält das Unternehmen die volle Kontrolle über den Prozess, was die geforderte Flexibilität ermöglicht, um auf Änderungen zu reagieren.


\subsubsection{Projektkosten}
\label{sec:Projektkosten}
%\begin{itemize}
%	\item Welche Kosten fallen bei der Umsetzung des Projekts im Detail an (\zB Entwicklung, Einführung/Schulung, Wartung)?
%\end{itemize}

\paragraph{Rechnung (verkürzt)}
Die Kosten für die Durchführung des Projekts setzen sich sowohl aus Personal-, als auch aus Ressourcenkosten zusammen.
Laut Tarifvertrag verdient ein Auszubildender im dritten Lehrjahr pro Monat \eur{900} Brutto.

\begin{eqnarray}
8 \mbox{ h/Tag} \cdot 220 \mbox{ Tage/Jahr} = 1.760 \mbox{ h/Jahr}\\
\eur{900}\mbox{/Monat} \cdot 12 \mbox{ Monate/Jahr} = \eur{10.800} \mbox{/Jahr}\\
\frac{\eur{10.800} \mbox{/Jahr}}{1760 \mbox{ h/Jahr}} \approx \eur{6,14}\mbox{/h}
\end{eqnarray}

Es ergibt sich also ein Stundenlohn von \eur{6,14}.
Die Durchführungszeit des Projekts beträgt 80 Stunden. Für die Nutzung von Ressourcen\footnote{Räumlichkeiten, Arbeitsplatzrechner \etc} wird
ein pauschaler Stundensatz von \eur{14} angenommen. Für die anderen Mitarbeiter wird pauschal ein Stundenlohn von \eur{25} angenommen.
Eine Aufstellung der Kosten befindet sich in Tabelle~\ref{tab:Kostenaufstellung} und sie betragen insgesamt \eur{1.806,20}.
\tabelle{Kostenaufstellung}{tab:Kostenaufstellung}{Kostenaufstellung.tex}


\subsubsection{Amortisationsdauer}
\label{sec:Amortisationsdauer}
%\begin{itemize}
%	\item Welche monetären Vorteile bietet das Projekt (\zB Einsparung von Lizenzkosten, Arbeitszeitersparnis, bessere Usability, Korrektheit)?
%	\item Wann hat sich das Projekt amortisiert?
%\end{itemize}
Das Produkt führt voraussichtlich zu einer Verkürzung der Vorbereitungsdauer eines Termins um etwa 20 Minuten.
Zusätzlich wird für das verteilen von allgemeinen Informationen pauschal 30 Minuten einkalkuliert.
Angenommen wird, dass durchschnittlich 10 Kunden zu Besuch kommen und sich 3 Mal pro Woche allgemeine Informationen ändern, während 1 Mitarbeiter dafür verantwortlich ist.
Unter Berücksichtigung dieser Annahmen ergibt sich folgende Berechnung.

\paragraph{Rechnung (verkürzt)}
Bei einem Zeitersparnis von 20 Minuten pro Termin und 30 Minuten für jede \ac{IV}, bei einem Benutzer und 44 Arbeitswochen im Jahr, wobei sich 10 Termine und 3 \ac{IV} pro Woche ereignen, ergibt sich folgendes Gesamtzeitersparnis.
\begin{eqnarray}
44 \mbox{ Woche/Jahr} \cdot (20 \mbox{ min/Termin} \cdot 10 \mbox{ Termin/Woche}) = 8800 \mbox{ min/Jahr} \approx 147 \mbox{ h/Jahr}\\
44 \mbox{ Woche/Jahr} \cdot (30 \mbox{ min/IV} \cdot 3 \mbox{ IV/Woche}) = 3960 \mbox{ min/Jahr} \approx 66 \mbox{ h/Jahr}
\end{eqnarray}

Dadurch ergibt sich eine jährliche Einsparung von 
\begin{eqnarray}
(147 \mbox{ h} + 66 \mbox{ h}) \cdot \eur{(25 + 14)}{\mbox{/h}} = \eur{8307}
\end{eqnarray}

Die Amortisationszeit beträgt also $\frac{\eur{Entwicklungskosten}}{\eur{Ersparnis}\mbox{/Jahr}} = \frac{1.806,20}{8307\mbox{/Jahr}} \approx 0,22 \mbox{ Jahre} \approx 11,5 \mbox{ Wochen}$.


%\subsection{Nutzwertanalyse}
%\label{sec:Nutzwertanalyse}
%\begin{itemize}
%	\item Darstellung des nicht-monetären Nutzens (\zB Vorher-/Nachher-Vergleich anhand eines Wirtschaftlichkeitskoeffizienten).
%\end{itemize}
%
%\paragraph{Beispiel}
%Ein Beispiel für eine Entscheidungsmatrix findet sich in Kapitel~\ref{sec:Architekturdesign}: \nameref{sec:Architekturdesign}.

\subsection{Nicht monetäre Vorteile}
\label{sec:NichtmonotaereVorteile}
Einer der positiven Aspekte, die die Durchführung dieses Projekts mit sich zieht, ist der gegenüber den Kunden und Anwendern des Moduls vermittelte Eindruck einer zukunfts- und kundenorientierten Entwicklung.
Wie bereits in der Projektbegründung erwähnt, können sich die Kunden und Besucher davon überzeugen, dass das Unternehmen sich stets weiterentwickelt.
Einen ähnlichen Effekt wird diese Umsetzung auf die Mitarbeiter haben.
Diese werden sehen, dass Ressourcen für die Erleichterung Ihres Arbeitsalltags aufgewendet werden.


\subsection{Anwendungsfälle}
\label{sec:Anwendungsfaelle}
%\begin{itemize}
%	\item Welche Anwendungsfälle soll das Projekt abdecken?
%	\item Einer oder mehrere interessante (!) Anwendungsfälle könnten exemplarisch durch ein Aktivitätsdiagramm oder eine \ac{EPK} detailliert beschrieben werden.
%\end{itemize}
%
Eine Übersicht über die verschiedenen Anwendungsfälle des Projekts im \Anhang{app:UseCase}.


\subsection{Qualitätsanforderungen}
\label{sec:Qualitaetsanforderungen}
%\begin{itemize}
%	\item Welche Qualitätsanforderungen werden an die Anwendung gestellt (\zB hinsichtlich Performance, Usability, Effizienz \etc (siehe \citet{ISO9126}))?
%\end{itemize}
Die Anwendung muss stabil und konsistent sein, um Ausfälle oder Betriebsunterbrechungen zu vermeiden.
Hierbei ist eine robuste Fehlerbehandlung entscheidend, ebenso wie die Fähigkeit, automatisch die zuletzt eingegebenen Informationen wiederherzustellen.
Das Design sollte ansprechend sein und sich nahtlos in das Corporate Design einfügen, um dem Image gerecht zu werden.
Darüber hinaus ist es wichtig, die Performance im Blick zu behalten, da die Anwendung über Electron.js lokal auf einem nicht leistungsstarken Mini-PC laufen wird.


\subsection{Lastenheft/Fachkonzept}
\label{sec:Lastenheft}
%\begin{itemize}
%	\item Auszüge aus dem Lastenheft/Fachkonzept, wenn es im Rahmen des Projekts erstellt wurde.
%	\item Mögliche Inhalte: Funktionen des Programms (Muss/Soll/Wunsch), User Stories, Benutzerrollen
%\end{itemize}
Im Rahmen des agilen Projekts werden die Anforderungen in Form eines Lastenhefts, Funktionen und User Stories erfasst, die die Grundlage für die Entwicklung der digitalen Gästebegrüßungs- und Infotafelanwendung bilden.


Die Funktionen umfassen sowohl wesentliche als auch optionale Aspekte.
Zu den unverzichtbaren Funktionen gehören die personalisierte Gästebegrüßung, die Anzeige von aktuellen Veranstaltungen und Terminen, die Integration von Live-Video-Streams sowie die Anzeige der aktuellen Uhrzeit.
Optionale Funktionen sind die Anpassung von Inhalten durch Mitarbeiter und das Bereitstellen von Hintergrund Animationen wie \zB einen Ken Burns Effekt.


Die User Stories repräsentieren die Bedürfnisse und Anwendungsfälle aus Sicht der verschiedenen Benutzer.

Ein Beispiel für eine User Story lautet: \("\)Als Gast möchte ich beim Betreten der Einrichtung freundlich begrüßt werden und relevante Informationen angezeigt bekommen.\("\)

Eine andere User Story ist: \("\)Als Mitarbeiter möchte ich in der Lage sein, die Inhalte der digitalen Infotafel einfach zu aktualisieren und anzupassen.\("\)


Ein Lastenheft findet sich im \Anhang{app:Lastenheft}.
