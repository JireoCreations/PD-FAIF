% !TEX root = ../Projektdokumentation.tex
\section{Dokumentation}
\label{sec:Dokumentation}

%\begin{itemize}
%	\item Wie wurde die Anwendung für die Benutzer/Administratoren/Entwickler dokumentiert (\zB Benutzerhandbuch, \acs{API}-Dokumentation)?
%	\item Hinweis: Je nach Zielgruppe gelten bestimmte Anforderungen für die Dokumentation (\zB keine IT-Fachbegriffe in einer Anwenderdokumentation verwenden, aber auf jeden Fall in einer Dokumentation für den IT-Bereich).
%\end{itemize}
%
%\paragraph{Beispiel}
%Ein Ausschnitt aus der erstellten Benutzerdokumentation befindet sich im \Anhang{app:BenutzerDoku}.
%Die Entwicklerdokumentation wurde mittels PHPDoc\footnote{Vgl. \cite{phpDoc}} automatisch generiert. Ein beispielhafter Auszug aus der Dokumentation einer Klasse findet sich im \Anhang{app:Doc}.
%\subsection{Projektdokumentation}
%\label{sec:Projektdokumentation}
Die vorliegende Projektdokumentation gibt einen Überblick über die Entstehung und Umsetzung des Projekts.
Sie ist in LaTeX10 verfasst worden, orientiert sich an vorhandenen Projektdokumentationen sowie an den Vorgaben der Prüfungsordnung und den offiziellen Richtlinien der \ac{IHK} zur Projektdokumentation.