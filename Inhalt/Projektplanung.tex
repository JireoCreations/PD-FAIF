% !TEX root = ../Projektdokumentation.tex
\section{Projektplanung} 
\label{sec:Projektplanung}


\subsection{Projektphasen}
\label{sec:Projektphasen}

%\begin{itemize}
%	\item In welchem Zeitraum und unter welchen Rahmenbedingungen (\zB Tagesarbeitszeit) findet das Projekt statt?
%	\item Verfeinerung der Zeitplanung, die bereits im Projektantrag vorgestellt wurde.
%\end{itemize}
Der Zeitraum für das Projekt startet am 26. Februar 2024 und endet am 12. April 2024, wobei es in Gleitzeit durchgeführt wird.
Während dieser Zeit wird es im Büro bearbeitet, um eine effektive Zusammenarbeit mit dem Stakeholder zu gewährleisten.

Tabelle~\ref{tab:Zeitplanung} zeigt eine grobe Zeitplanung.
\tabelle{Zeitplanung}{tab:Zeitplanung}{ZeitplanungKurz}\\
Eine detailliertere Zeitplanung findet sich im \Anhang{app:Zeitplanung}.


\subsection{Abweichungen vom Projektantrag}
\label{sec:AbweichungenProjektantrag}
%\begin{itemize}
%	\item Sollte es Abweichungen zum Projektantrag geben (\zB Zeitplanung, Inhalt des Projekts, neue Anforderungen), müssen diese explizit aufgeführt und begründet werden.
%\end{itemize}
In diesem Projekt wurden keine automatisierten Tests eingesetzt, da die Anwendung eine einfache Funktionalität aufweist, die primär auf das Auslesen von Kalendereinträgen ausgerichtet ist.
Die Komplexität der Funktionalität sowie die Anforderungen an die Robustheit und Skalierbarkeit der Anwendung sind nicht hoch genug, um den Einsatz automatisierter Tests zu rechtfertigen.
Zudem ist der Aufwand für die Implementierung automatisierter Tests im Vergleich zum Nutzen nicht verhältnismäßig.
Stattdessen wird auf manuelle Tests gesetzt, um sicherzustellen, dass die grundlegende Funktionalität der Anwendung ordnungsgemäß funktioniert und den Anforderungen entspricht.


\subsection{Ressourcenplanung}
\label{sec:Ressourcenplanung}

%\begin{itemize}
%	\item Detaillierte Planung der benötigten Ressourcen (Hard-/Software, Räumlichkeiten \usw).
%	\item \Ggfs sind auch personelle Ressourcen einzuplanen (\zB unterstützende Mitarbeiter).
%\end{itemize}
Es werden ein Arbeitsgerät, eine stabile Internetverbindung sowie eine geeignete Entwicklungsumgebung benötigt, hier wurde dafür \ac{VS} Code als \ac{IDE} verwendet.
Da die Anwendung auf \ac{MS} Graph \ac{API} zugreift, ist der Zugang zu Azure \ac{AD} unerlässlich.
Zusätzlich wird im Büro ein Fernseher installiert, um die Anwendung zu ermöglichen.
Für die Bedienung wird eine Bluetooth-Tastatur, Maus und ein Mini-PC bereitgestellt.

Es werden Git für das \ac{SV} und \ac{GL} zur effektiven und dokumentierte Verwaltung des Quellcodes genutzt.
Außerdem wird Jira in diesem Projekt verwendet und dadurch entsteht eine nahtlose Verbindung zwischen Projektverwaltung und Entwicklung.
Der Author kann direkt aus Jira-Tickets heraus an den entsprechenden Codeänderungen arbeiten und den Fortschritt der Aufgaben verfolgen.


\subsection{Entwicklungsprozess}
\label{sec:Entwicklungsprozess}
%\begin{itemize}
%	\item Welcher Entwicklungsprozess wird bei der Bearbeitung des Projekts verfolgt (\zB Wasserfall, agiler Prozess)?
%\end{itemize}
Das Projekt wird in einem agilen Arbeitsumfeld durchgeführt, wobei eine enge Zusammenarbeit mit dem Stakeholder gepflegt wird.
Tägliche Meetings sind angesetzt, um den aktuellen Entwicklungsstand zu besprechen und sich gegenseitig über neue oder geänderte Anforderungen abzustimmen.