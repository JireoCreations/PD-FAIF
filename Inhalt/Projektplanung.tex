% !TEX root = ../Projektdokumentation.tex
\section{Projektplanung} 
\label{sec:Projektplanung}


\subsection{Projektphasen}
\label{sec:Projektphasen}

%\begin{itemize}
%	\item In welchem Zeitraum und unter welchen Rahmenbedingungen (\zB Tagesarbeitszeit) findet das Projekt statt?
%	\item Verfeinerung der Zeitplanung, die bereits im Projektantrag vorgestellt wurde.
%\end{itemize}
Das Projekt startet am 26. Februar 2024 und endet am 12. April 2024, wobei es in Teilzeit durchgeführt wird.
Während dieser Zeit wird es im Büro bearbeitet, um eine effektive Zusammenarbeit mit Mark Barrenscheen zu gewährleisten.
Die Nähe ermöglicht es, Änderungen schnell zu besprechen und umzusetzen.

Tabelle~\ref{tab:Zeitplanung} zeigt eine grobe Zeitplanung.
\tabelle{Zeitplanung}{tab:Zeitplanung}{ZeitplanungKurz}\\
Eine detailliertere Zeitplanung findet sich im \Anhang{app:Zeitplanung}.


\subsection{Abweichungen vom Projektantrag}
\label{sec:AbweichungenProjektantrag}
%\begin{itemize}
%	\item Sollte es Abweichungen zum Projektantrag geben (\zB Zeitplanung, Inhalt des Projekts, neue Anforderungen), müssen diese explizit aufgeführt und begründet werden.
%\end{itemize}
In diesem Projekt wurden keine automatisierten Tests eingesetzt, da die Anwendung eine einfache Funktionalität aufweist, die primär auf das Auslesen von Kalendereinträgen ausgerichtet ist.
Die Komplexität der Funktionalität sowie die Anforderungen an die Robustheit und Skalierbarkeit der Anwendung waren nicht hoch genug, um den Einsatz automatisierter Tests zu rechtfertigen.
Zudem wäre der Aufwand für die Implementierung automatisierter Tests im Vergleich zum potenziellen Nutzen möglicherweise nicht verhältnismäßig gewesen.
Stattdessen wurde auf manuelle Tests gesetzt, um sicherzustellen, dass die grundlegende Funktionalität der Anwendung ordnungsgemäß funktioniert und den Anforderungen entspricht.


\subsection{Ressourcenplanung}
\label{sec:Ressourcenplanung}

%\begin{itemize}
%	\item Detaillierte Planung der benötigten Ressourcen (Hard-/Software, Räumlichkeiten \usw).
%	\item \Ggfs sind auch personelle Ressourcen einzuplanen (\zB unterstützende Mitarbeiter).
%\end{itemize}
Es werden ein Arbeitsgerät, eine stabile Internetverbindung sowie eine geeignete Entwicklungsumgebung benötigt, hier wurde dafür VSCode als IDE verwendet.
Da die Anwendung auf Microsoft Azure Active Directory zugriff, war der Zugang zu Azure AD unerlässlich.
Zusätzlich wurde im Büro ein Fernseher installiert, um Präsentationen und Demos der Anwendung zu ermöglichen.
Für die Remote-Arbeit wurden Remote-Tastatur, Maus und ein Mini-PC bereitgestellt, um eine reibungslose Zusammenarbeit zwischen den Teammitgliedern zu gewährleisten.
Diese sorgfältige Ressourcenplanung trug dazu bei, dass das Team effizient arbeiten konnte und Mark Barrenscheen sowie andere Nutzer die Anwendung produktiv nutzen konnten.


\subsection{Entwicklungsprozess}
\label{sec:Entwicklungsprozess}
\begin{itemize}
	\item Welcher Entwicklungsprozess wird bei der Bearbeitung des Projekts verfolgt (\zB Wasserfall, agiler Prozess)?
\end{itemize}
