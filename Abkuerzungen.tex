% !TEX root = Projektdokumentation.tex

% Es werden nur die Abkürzungen aufgelistet, die mit \ac definiert und auch benutzt wurden. 
%
% \acro{VERSIS}{Versicherungsinformationssystem\acroextra{ (Bestandsführungssystem)}}
% Ergibt in der Liste: VERSIS Versicherungsinformationssystem (Bestandsführungssystem)
% Im Text aber: \ac{VERSIS} -> Versicherungsinformationssystem (VERSIS)

% Hinweis: allgemein bekannte Abkürzungen wie z.B. bzw. u.a. müssen nicht ins Abkürzungsverzeichnis aufgenommen werden
% Hinweis: allgemein bekannte IT-Begriffe wie Datenbank oder Programmiersprache müssen nicht erläutert werden,
%          aber ggfs. Fachbegriffe aus der Domäne des Prüflings (z.B. Versicherung)

% Die Option (in den eckigen Klammern) enthält das längste Label oder
% einen Platzhalter der die Breite der linken Spalte bestimmt.
\begin{acronym}[WWWWW]
	\acro{AD}{Active Directory}
	\acro{AG}{Aktiengesellschaft}
	\acro{API}{Application Programming Interface}
	\acro{CSS}{Cascading Style Sheets}
	\acro{CSTx ES}{CSTx Enterprise Solutions}
	\acro{CSTx SE}{CSTx Software Engineering}
	\acro{CP}{Container-Presentational}
	\acro{GL}{GitLab}
	\acro{GmbH}{Gesellschaft mit beschränkter Haftung}
	\acro{HTML}{Hypertext Markup Language}
	\acro{IDE}{Integrated Development Environment}
	\acro{IHK}{Industrie- und Handelskammer}
	\acro{IPC}{Inter-Process Communication}
	\acro{IV}{Informationsverteilung}
	\acro{JS}{JavaScript}
	\acro{JSX}{JavaScript XML}
	\acro{MS}{Microsoft}
	\acro{MSAL}{Microsoft Authentication Library}
	\acro{Source Versioning}
	\acro{TS}{TypeScript}
	\acro{TSX}{TypeScript XML}
	\acro{UI}{User Interface}
	\acro{UML}{Unified Modeling Language}
	\acro{VS}{Visual Studio}
\end{acronym}
