\subsection{Lastenheft (Auszug)}
\label{app:Lastenheft}
Es folgt ein Auszug aus dem Lastenheft mit Fokus auf die Anforderungen:

Die Anwendung muss folgende Anforderungen erfüllen: 
\begin{enumerate}[itemsep=0em,partopsep=0em,parsep=0em,topsep=0em]
\item Authentifizierung
	\begin{enumerate}
	\item Die Anwendung muss in der Lage sein, mit Azure Active Directory (Azure AD) zu integrieren, um die Authentifizierung von Benutzern zu ermöglichen.
	\item Die Anwendung soll den Azure AD Code Authentication Flow verwenden, um die Authentifizierung von Benutzern zu ermöglichen. Dies beinhaltet den Austausch von Autorisierungscodes gegen Zugriffstoken und Refresh-Token.
	\item Die Anwendung braucht die nötigen Berechtigungen, um auf den vom Benutzer eingeloggten Kalender über Azure AD und dem zugehörigen Tenant zuzugreifen.
	\end{enumerate}
\item Darstellung der Daten
	\begin{enumerate}
	\item Die Anwendung muss eine Liste aller noch folgenden Termine für den aktuellen Tag anzeigen.
	\item Vergangene Termine müssen 30 Minuten nach beendigung weiterhin angezeigt werden.
	\item Jeder Besucher soll über die Oberfläche mit folgenden Informationen begrüßt werden.:
	\begin{enumerate}
		\item Dem vollständigen Namen.
		\item Der angehörigkeit des Besuchers.
		\item Der Kontaktperson oder Organisator des Termins.
		\item \Ggfs Zusätzlicher oder relevanter Informationen zum Termin.
	\end{enumerate}
	\item Die Anwendung muss eine Laufschrift am unteren Bildschirmrand für allgemeine Informationen bereitstellen.
	\item Die Anwendung muss einen für die wartenden Besucher bereitgestellten Live-Stream von \zB den öffentlichen Fernsehprogrammen darstellen.
	\item Die aktuelle Uhrzeit und das zu verzeichnende Datum muss angezeigt werden.
	\end{enumerate}
\item Sonstige Anforderungen
	\begin{enumerate}
	\item Die Anwendung soll als gebaute Executable ausführbar sein.
	\item Die Anwendung soll über den aktuellsten 0 Uhr Kalender-Eintrag konfiguriert werden können. Dazu zählen folgende Konfigurationsmöglichkeiten.:
	\begin{enumerate}
		\item Das setzen des Hintergrundbildes der Anwendung.
		\item Das setzen des Texts für die Laufschrift.
	\end{enumerate}
	\end{enumerate}
\end{enumerate}

