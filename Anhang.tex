% !TEX root = Projektdokumentation.tex
\section{Anhang}
\subsection{Detaillierte Zeitplanung}
\label{app:Zeitplanung}

\tabelleAnhang{ZeitplanungKomplett}

\subsection{Lastenheft (Auszug)}
\label{app:Lastenheft}
Es folgt ein Auszug aus dem Lastenheft mit Fokus auf die Anforderungen:

Die Anwendung muss folgende Anforderungen erfüllen: 
\begin{enumerate}[itemsep=0em,partopsep=0em,parsep=0em,topsep=0em]
\item Authentifizierung
	\begin{enumerate}
	\item Die Anwendung muss in der Lage sein, mit \ac{MS} Azure \ac{AD} zu interagieren, um die Authentifizierung von Benutzern zu ermöglichen.
	\item Die Anwendung soll den \ac{MS} Azure \ac{AD} Authentication Code Flow verwenden, um die Authentifizierung von Benutzern zu ermöglichen. Dies beinhaltet den Austausch von Autorisierungscodes gegen Access-Token und Refresh-Token.
	\item Die Anwendung braucht die nötigen Berechtigungen, um auf den vom Benutzer eingeloggten Kalender über \ac{MS} Azure \ac{AD} und dem zugehörigen Tenant zuzugreifen.
	\end{enumerate}
\item Darstellung der Daten
	\begin{enumerate}
	\item Die Anwendung muss eine Liste aller noch folgenden Termine für den aktuellen Tag anzeigen.
	\item Vergangene Termine müssen 30 Minuten nach Beendigung weiterhin angezeigt werden.
	\item Jeder Besucher soll über die Oberfläche mit folgenden Informationen begrüßt werden.:
	\begin{enumerate}
		\item Dem vollständigen Namen.
		\item Der Firmenzugehörigkeit des Besuchers.
		\item Der Kontaktperson oder Organisator des Termins.
		\item \Ggfs zusätzlicher oder relevanter Informationen zum Termin.
	\end{enumerate}
	\item Die Anwendung muss eine Laufschrift am unteren Bildschirmrand für allgemeine Informationen bereitstellen.
	\item Die Anwendung muss einen für die wartenden Besucher bereitgestellten Live-Stream von \zB den öffentlichen Fernsehprogrammen darstellen.
	\item Die aktuelle Uhrzeit und Datum muss angezeigt werden.
	\end{enumerate}
\item Sonstige Anforderungen
	\begin{enumerate}
	\item Die Anwendung soll als gebaute Executable ausführbar sein.
	\item Die Anwendung soll über den aktuellsten 0 Uhr Kalender-Eintrag konfiguriert werden können. Dazu zählen folgende Konfigurationsmöglichkeiten.:
	\begin{enumerate}
		\item Das setzen des Hintergrundbildes der Anwendung.
		\item Das setzen des Texts für die Laufschrift.
	\end{enumerate}
	\end{enumerate}
\end{enumerate}


\clearpage

\subsection{Verwendete Ressourcen}
\label{app:VerwendeteRessourcen}

\paragraph{Hardware}
\begin{itemize}
	\item MacBook Pro 13 Zoll
	\item Docking Station
	\item 2 Monitore
	\item Bluetooth Headset
	\item Bluetooth Maus und Tastatur
\end{itemize}

\paragraph{Software}
\begin{itemize}
	\item Betriebssystem: macOS Sonoma 14.1.2
	\item Entwicklungsumgebung: \ac{VS} Code mit Erweiterungen
	\item Programmiersprache \ac{TS} 4.5.5
	\item Auszeichnungssprache: \ac{HTML} und \ac{TSX}
	\item Paketmanager Frontend: npm
	\item Diagrammframework: PlantUML\footnote{\ac{UML}} und Draw.io
	\item Stylesheetsprache: \ac{CSS} und Tailwind \ac{CSS}
	\item Versionierung: git/GitLab
	\item Zielplatform: Webbrowser (Chromium)
	\item Meetings: Microsoft Teams und Büro Räumlichkeiten
	\item Unternehmensplatform: Microsoft 365
	\item Mockups: Excalidraw
	\item Dokumentation: LaTeX
	\item Dokumentationseditor: \ac{VS} Code mit Erweiterungen
\end{itemize}

\paragraph{Personal}
\begin{itemize}
	\item Stakeholder für Betreuung, Codereview und Qualitätssicherung, sowohl für Festlegung der Anforderungen an das Projekt und Projektabnahme.
	\item Anwender des Produkts.
\end{itemize}

\subsection{Use Case-Diagramm}
\label{app:UseCase}
\begin{figure}[htb]
\centering
\includegraphicsKeepAspectRatio{use_case.png}{0.7}
\caption{Use Case-Diagramm}
\end{figure}

%\input{Anhang/AnhangPflichtenheft.tex}
%
%\subsection{Datenbankmodell}
%\label{app:Datenbankmodell}
%ER-Modelle kann man auch direkt mit \LaTeX{} zeichnen, siehe \zB \url{http://www.texample.net/tikz/examples/entity-relationship-diagram/}.
%\begin{figure}[htb]
%\centering
%\includegraphicsKeepAspectRatio{database.pdf}{1}
%\caption{Datenbankmodell}
%\end{figure}
\clearpage

\subsection{Oberflächenentwürfe}
\label{app:Entwuerfe}
\begin{figure}[htb]
\centering
\includegraphicsKeepAspectRatio{wireframe-standby.png}{0.7}
\caption{Wireframe der Standby-Ansicht}
\end{figure}

\begin{figure}[htb]
\centering
\includegraphicsKeepAspectRatio{wireframe-fullscreen.png}{0.7}
\caption{Wireframe der Fullscreen-Ansicht}
\end{figure}

\begin{figure}[htb]
\centering
\includegraphicsKeepAspectRatio{wireframe-event.png}{0.7}
\caption{Wireframe der Ansicht mit einem Termin}
\end{figure}

\begin{figure}[htb]
    \centering
    \includegraphicsKeepAspectRatio{wireframe-mehr-events.png}{0.7}
    \caption{Wireframe der Ansicht mit mehreren Terminen}
\end{figure}

\clearpage
\subsection{Produktfotos der Anwendung}
\label{Produktfotos}
\begin{figure}[htb]
\centering
\includegraphicsKeepAspectRatio{Produktfoto_1.jpg}{1}
\caption{Ansicht im Standby}
\end{figure}
\clearpage
\begin{figure}[htb]
    \centering
    \includegraphicsKeepAspectRatio{Produktfoto_2.jpg}{1}
    \caption{Vollbildansicht vom Live-Video}
\end{figure}
\clearpage
\begin{figure}[htb]
    \centering
    \includegraphicsKeepAspectRatio{Produktfoto_3.jpg}{1}
    \caption{Ansicht mit einem Termin}
\end{figure}
\clearpage
\begin{figure}[htb]
    \centering
    \includegraphicsKeepAspectRatio{Produktfoto_4.jpg}{1}
    \caption{Ansicht mit mehreren Terminen}
\end{figure}
\clearpage

%\input{Anhang/AnhangDoc.tex}
%\clearpage
%\input{Anhang/AnhangTest.tex}

\subsection{Main-Komponente Logik-Loop}
\label{app:MAINLOOP}
Main Logic-Loop für das Pulling der Kalendereinträge
\lstinputlisting[language=typescript, caption={Main-Komponente Logik-Loop}]{Listings/main.ts}
\clearpage

\subsection{Main-Komponente IPC}
\label{app:MAINIPC}
Main \ac{IPC} für Kommunikation zwischen Renderer und Main
\lstinputlisting[language=typescript, caption={Main-Komponente IPC}]{Listings/ipcMain.ts}
\clearpage

\subsection{Projektstruktur}
\label{app:Projektstruktur}
\begin{figure}[htb]
\centering
\includegraphicsKeepAspectRatio{Projektstruktur.drawio.png}{1}
\caption{Projektstruktur}
\end{figure}
\clearpage

%\input{Anhang/AnhangBenutzerDoku.tex}
